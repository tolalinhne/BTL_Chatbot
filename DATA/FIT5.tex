\documentclass[12pt]{article}
\usepackage{geometry}
\usepackage{amsmath}
\usepackage{amssymb}
\usepackage{enumitem}
\usepackage{fancyhdr}
\usepackage{graphicx}
\usepackage{tikz}
\usepackage[utf8]{vietnam}
\usetikzlibrary{trees}
\pagestyle{fancy}
\usepackage[vietnamese]{babel}
\usepackage{float} % Thêm để dùng [H]
\lhead{B23DCCN488}
\chead{Phạm Mai Linh}
\rhead{PTIT ELE1319 HW01}
%
\lfoot{11/05/2025}
%\chead{Austin Bossart - Homework Number}
\rfoot{EIC \& DSP Lab.}

\begin{document}
%% =================
%%		Nội dung minh họa cách làm bài bắt đầu từ đây
%% =================

\textbf{Câu 1}: Tính entropy của nguồn rời rạc không nhớ
\begin{equation*}
    X = \begin{pmatrix}
        x_1 & x_2 & x_3 & x_3 \\
        \frac{1}{4} & \frac{1}{8} & \frac{1}{8} & \frac{1}{2}
    \end{pmatrix}
\end{equation*}

\textbf{Bài làm}: Áp dụng công thức định nghĩa Entropy của nguồn rời rạc không nhớ:
$$H(X)=-\sum_{k=1}^{N}p(x_k)log(p(x_k))$$

Từ dữ liệu đã cho, ta có $N=4$, triển khai công thức ta có:
\begin{align*}
    H(X) &= -\sum_{k=1}^{4}p(x_k)log(p(x_k))\\
    &= -\big(p(x_1)log(p(x_1))+p(x_2)log(p(x_2))+p(x_3)log(p(x_3))+p(x_4)log(p(x_4)) \big)
\end{align*}

Thay giá trị của $p(x_k)$ ($k=\bar{1,4}$) ta được:
\begin{align*}
    H(X) &= -\big(\frac{1}{4}log(\frac{1}{4})+\frac{1}{8}log(\frac{1}{8})+\frac{1}{8}log(\frac{1}{8})+\frac{1}{2}log(\frac{1}{2})) \big) \\
    &= 1.75 \text{ [bit] }
\end{align*}


\newpage
\textbf{Câu 2}: Chứng minh rằng $H(X)\le log_2(|X|)$.

\textbf{Bài làm}: Để đơn giản, đặt $N\triangleq |X|$ và $log_2\triangleq log$.

Xét biểu thức:
$$A = H(X)-log(N)$$
Áp dụng định nghĩa entropy của một nguồn rời rạc, thay vào công thức ta có:
$$A = -\sum_{k=1}^Np(x_k)log(p(x_k))-log(N)$$
Theo tiên đề xác suất,
$$\sum_{k=1}^Np(x_k)=1$$
Ta có thể viết:
$$log(N) = (log(N))\sum_{k=1}^Np(x_k)=\sum_{k=1}^Np(x_k)log(N)$$
Thay vào biểu thức $A$, ta có:
$$A = -\sum_{k=1}^Np(x_k)log(p(x_k))-\sum_{k=1}^Np(x_k)log(N)$$
Cả hai số hạng đều tính $\sum_{k=1}^N$ nên có thể viết chung thành:
$$A=-\sum_{k=1}^N\big(p(x_k))log(p(x_k))+p(x_k)log(N)\big)$$
Biểu thức trong tổng có thể đặt nhân tử chung $p(x_k)$ ta được:
$$A=-\sum_{k=1}^Np(x_k)\big(log(p(x_k))+log(N)\big)$$
Tiếp tục áp dụng tính chất toán học $log_ax+log_ay = log_a(xy)$, và $-log_a(x)=log_a(\frac{1}{x})$
biểu thức $A$ có thể viết thành:
$$A = \sum_{k=1}^Np(x_k)log(\frac{1}{Np(x_k)})$$
Theo kết quả toán sơ cấp, $log(\frac{1}{y})\le 1-\frac{1}{y}$; áp dụng với $y=Np(x_k)$ và kết quả vừa đề cập cho biểu thức $A$ được:
$$A\le \sum_{k=1}^Np(x_k)(1-\frac{1}{Np(x_k)})$$
Dễ thấy, biểu thức vế phải bằng:
$$\sum_{k=1}^Np(x_k)-\sum_{k=1}^N\frac{1}{N}=0$$
Do đó, $A\le 0$ suy ra đpcm.

\newpage
\textbf{Câu 3}: Cho tin $x_i$ có xác suất là $p(x_i)=0.5$, lượng tin riêng của tin này bằng các đại lượng nào dưới đây:

 \begin{enumerate}[label=\Alph*.]
    \item $4$ bit.
    \item (CORRECT) $1$ bit.
    \item $\frac{1}{4}$ bit.
    \item $2$ bit.
 \end{enumerate}


\textbf{Giải}: Theo công thức định nghĩa, lượng tin của một tin xuất hiện với xác suất $p(x_i)$ bằng $I(x_i)=-log_2(p(x_i))$, thay xác suất đã cho $p(x_i)=0.5$ vào ta có:

$$ I(x_i) = - log_2(0.5) = 1 [bit]$$

Do đó, đáp án là \textbf{B}

%% =================
%%		Nội dung minh họa cách làm kết thúc ở đây
%% =================
%% =================
%%		Nội dung bài làm sẽ bắt đầu từ đây
%% =================
\newpage

\textbf{Câu 1 (2.5 điểm):} Một nguồn nhị phân độc lập với phân bố xác suất nguồn là 0{,}25 và 0{,}75 được truyền trên kênh nhị phân đối xứng với xác suất chuyển sai $p = 0{,}01$. Tính các đại lượng $H(X|Y)$ và $I(X;Y)$. 

\[
\textbf{Bài làm:}
\]

\[
\begin{tikzpicture}[>=latex, thick]

% Các nút trái (đầu vào)
\node[left] (x1) at (0,3) {$x_1 = 0,25 $};
\node[left] (x2) at (0,0) {$x_2 = 0,75 $};

% Các nút phải (đầu ra)
\node[right] (y1) at (6,3) {$y_1$};
\node[right] (y2) at (6,0) {$y_2$};

% Các mũi tên từ 0
\draw[->] (x1) -- (y1) node[midway, above] {};
\draw[->] (x1) -- (y2) node[midway, sloped, below] {$p$};

% Các mũi tên từ 1
\draw[->] (x2) -- (y1) node[midway, sloped, above] {$p$};
\draw[->] (x2) -- (y2) node[midway, sloped, below] {};
\end{tikzpicture}
\]

Có: 
\[
\begin{aligned}
p(y_1|x_1) &= p(y_2|x_2) = 0{,}99 \\
p(y_1|x_2) &= p(y_2|x_1) = 0{,}01 \\
p(x_1 y_1) &= p(x_1) p(y_1|x_1) = 0{,}25 \cdot 0{,}99
\end{aligned}
\]

Ta có: 
\[
\begin{aligned}
H(X|Y) = -[ &p(x_1 y_1)\log p(x_1 |y_1) + p(x_1 y_2)\log p(x_1 |y_2) \\
&+ p(x_2 y_1)\log p(x_2 |y_1) + p(x_2 y_2)\log p(x_2 |y_2)]
\end{aligned}
\]

\[
p(x_1 y_1) = p(x_1) p(y_1|x_1) = 0{,}25 \cdot 0{,}99 = p(y_1) p(x_1|y_1)
\]

Mà:
\[
p(y_1) = p(x_1 y_1) + p(x_2 y_1) = 0{,}25 \cdot 0{,}99 + 0{,}75 \cdot 0{,}01 \approx 0{,}255 
\Rightarrow p(y_2) = 0{,}745
\]

\[
\Rightarrow p(x_1|y_1) = \frac{33}{34}, \quad p(x_2|y_1) = \frac{1}{34}
\]

\[
p(x_1 y_2) = p(x_1) p(y_2|x_1) = 0{,}25 \cdot 0{,}01 = p(y_2) p(x_1|y_2)
\]

\[
\Rightarrow p(x_1|y_2) = \frac{0{,}25 \cdot 0{,}01}{0{,}745} = \frac{1}{298}, \quad
p(x_2|y_2) = \frac{297}{298}
\]

\[
\Rightarrow H(X|Y) = - \left[
0{,}2475 \cdot \log \frac{33}{34} +
0{,}0025 \cdot \log \frac{1}{298} +
0{,}0075 \cdot \log \frac{1}{34} +
0{,}7425 \cdot \log \frac{297}{298}
\right] \approx 0{,}073\ \text{(bit)}
\]

\[
\begin{aligned}
H(X) &= - p(x_1)\log p(x_1) - p(x_2)\log p(x_2) \\
&= -0{,}25 \cdot \log 0{,}25 - 0{,}75 \cdot \log 0{,}75 = 0{,}811\ \text{(bit)}
\end{aligned}
\]

\[
\Rightarrow I(X;Y) = H(X) - H(X|Y) = 0{,}811 - 0{,}073 = 0{,}738\ \text{(bit)}
\]

\newpage
\textbf{Câu 2 (2.1):} Cho nguồn rời rạc $\alpha = \left( 
\begin{array}{ccccccc}
a & b & c & d & e & f & g \\
0{,}01 & 0{,}24 & 0{,}05 & 0{,}2 & 0{,}47 & 0{,}01 & 0{,}02
\end{array}
\right)$

a) Tính entropy của nguồn $\alpha$.

b) Không cần tính, em hãy cho biết tin nào trong nguồn này chứa nhiều thông tin nhất và giải thích tại sao.

\[
\textbf{Bài làm}
\]

\textbf{a)}

\[
H(\alpha) = - p(a)\log p(a) - p(b)\log p(b) - \dots - p(g)\log p(g) \approx 1{,}932 \text{ (bit)}
\]

\textbf{b)}

\[
I(x_i) = -\log x_i
\Rightarrow p \text{ càng nhỏ, } I \text{ càng lớn}
\]

Có: $p(a) = p(f) = 0{,}01$ là nhỏ nhất 
$\Rightarrow I(a) = I(f)$ là lớn nhất: 
\[
I(a) = I(f) = - \log 0{,}01 \approx 6{,}64 \text{ (bit)}
\]

\newpage
\textbf{Câu 3 (4.3):} Cho sơ đồ kênh trong đó nguồn tín hiệu phát phát gồm $X=(x1,x2)$. Biết xác suất phát các tín hiệu $p(x_1)=p(x_2)=0,5$
\[
\begin{tikzpicture}[>=latex, thick]

% Các nút đầu vào (trái)
\node[left] (x1) at (0,3) {$x_1$};
\node[left] (x2) at (0,0) {$x_2$};

% Các nút đầu ra (phải)
\node[right] (y1) at (5,3) {$y_1$};
\node[right] (y3) at (5,1.5) {$y_3$};
\node[right] (y2) at (5,0) {$y_2$};

% Các mũi tên từ x1
\draw[->] (x1) -- (y1) node[midway, above] {$\alpha$};
\draw[->] (x1) -- (y3) node[midway, sloped, below] {$1 - \alpha$};

% Các mũi tên từ x2
\draw[->] (x2) -- (y2) node[midway, below] {$\alpha$};
\draw[->] (x2) -- (y3) node[midway, sloped, above] {$1 - \alpha$};

\end{tikzpicture}
\]

a) Hãy tính $H(X)$

b) Hãy tính $p(X=x_n,Y=y_m)$ với $n=1,2$ và $m=1,2,3$ để từ đó tính $H(X,Y)$ dưới dạng hàm số của $\alpha$

c) Hãy tính $p(Y=y_m)$, với $m=1,2,3$, để từ đó tính $H(Y)$ dưới dang hàm số của $\alpha$ 

d) Hãy tính $I(X,Y)$ dưới hàng hàm số của $\alpha$ ? Hãy xác định giá trị của $\alpha$ khi $I(X,Y)$ đạt giá trị cực đại và khi $I(X,Y)$ đạt giá trị cực tiểu? Hãy cho biết ý nghĩa trực quan của các kênh với các giá trị cực trị của $I(X,Y)$ ?

\[
\textbf{Bài làm}
\]

\textbf{a) Tính $H(X)$:}

\[
H(X) = - \sum_{i=1}^{2} p(x_i) \log_2 p(x_i) = -2 \cdot \frac{1}{2} \log_2 \left( \frac{1}{2} \right) = \log_2 2 = 1
\]

\textbf{b) Tính $p(x_n, y_m)$ và $H(X,Y)$:}

\[
\begin{aligned}
p(x_1, y_1) &= p(x_1) \cdot p(y_1|x_1) = \frac{1}{2} \cdot \alpha = \frac{\alpha}{2} \\
p(x_1, y_3) &= \frac{1}{2} \cdot (1 - \alpha) = \frac{1 - \alpha}{2} \\
p(x_1, y_2) &= 0 \\
p(x_2, y_2) &= \frac{1}{2} \cdot \alpha = \frac{\alpha}{2} \\
p(x_2, y_3) &= \frac{1}{2} \cdot (1 - \alpha) = \frac{1 - \alpha}{2} \\
p(x_2, y_1) &= 0
\end{aligned}
\]

Từ đó:

\[
\begin{aligned}
H(X,Y) &= -\sum_{i=1}^{2} \sum_{j=1}^{3} p(x_i, y_j) \log_2 p(x_i, y_j) \\
&= -\left( \frac{\alpha}{2} \log_2 \frac{\alpha}{2} + \frac{1 - \alpha}{2} \log_2 \frac{1 - \alpha}{2} + \frac{\alpha}{2} \log_2 \frac{\alpha}{2} + \frac{1 - \alpha}{2} \log_2 \frac{1 - \alpha}{2} \right) \\
&= -\left( \alpha \log_2 \frac{\alpha}{2} + (1 - \alpha) \log_2 \frac{1 - \alpha}{2} \right)
\end{aligned}
\]

\textbf{c) Tính $p(y_j)$ và $H(Y)$:}

\[
\begin{aligned}
p(y_1) &= p(x_1, y_1) + p(x_2, y_1) = \frac{\alpha}{2} + 0 = \frac{\alpha}{2} \\
p(y_2) &= 0 + \frac{\alpha}{2} = \frac{\alpha}{2} \\
p(y_3) &= \frac{1 - \alpha}{2} + \frac{1 - \alpha}{2} = 1 - \alpha
\end{aligned}
\]

\[
\begin{aligned}
H(Y) &= - \sum_{j=1}^{3} p(y_j) \log_2 p(y_j) \\
&= - \left( \frac{\alpha}{2} \log_2 \frac{\alpha}{2} + \frac{\alpha}{2} \log_2 \frac{\alpha}{2} + (1 - \alpha) \log_2 (1 - \alpha) \right) \\
&= - \left( \alpha \log_2 \frac{\alpha}{2} + (1 - \alpha) \log_2 (1 - \alpha) \right)
\end{aligned}
\]

\textbf{d) Tính $I(X;Y)$ và phân tích cực trị:}

\[
\begin{aligned}
I(X;Y) &= H(X) + H(Y) - H(X,Y) \\
&= 1 - \left[ \alpha \log_2 \frac{\alpha}{2} + (1 - \alpha) \log_2 (1 - \alpha) \right] - \left[ -\left( \alpha \log_2 \frac{\alpha}{2} + (1 - \alpha) \log_2 \frac{1 - \alpha}{2} \right) \right] \\
&= 1 - \alpha \log_2 \frac{\alpha}{2} - (1 - \alpha) \log_2 (1 - \alpha) + \alpha \log_2 \frac{\alpha}{2} + (1 - \alpha) \log_2 \frac{1 - \alpha}{2} \\
&= 1 + (1 - \alpha) \left( \log_2 \frac{1 - \alpha}{2} - \log_2 (1 - \alpha) \right) \\
&= 1 + (1 - \alpha) \log_2 \frac{1 - \alpha}{2(1 - \alpha)} = 1 + (1 - \alpha) \log_2 \frac{1}{2} = 1 - (1 - \alpha)
\end{aligned}
\]

Vậy:

\[
I(X;Y) = \alpha
\]

Giá trị cực đại của $I(X;Y)$ là $\alpha = 1$, khi đó tín hiệu đầu ra hoàn toàn xác định bởi đầu vào.

Giá trị cực tiểu là $\alpha = 0$, khi đó toàn bộ tín hiệu đầu ra là $y_3$ — tức kênh hoàn toàn nhiễu, không mang thông tin.

\textbf{Ý nghĩa trực quan:}  
- Khi $\alpha = 1$: kênh không có nhiễu, đầu vào được truyền chính xác đến đầu ra $\Rightarrow$ kênh lý tưởng.  
- Khi $\alpha = 0$: mọi tín hiệu đều bị biến thành $y_3$, không thể phân biệt được đầu vào nào được gửi đi $\Rightarrow$ kênh vô ích.

\newpage
\textbf{Câu 4 (2.19)}: Tín hiệu thoại có băng tần W=3,4kHz

a. Tính khả năng thông qua của kênh với điều kiện SNR=30dB

b. Tính SNR tối thiểu cần thiết để kênh có thể truyền tín hiệu thoại số có tốc độ 4800bps

\[
\textbf{Bài làm}
\]

a) Có W=3400(Hz)=F
\[
SNR=10 \log_{10} \left( \frac{S}{N} \right)=30 (dB)
\]
\[
\Rightarrow \frac{S}{N} = 10^3
\]

Khả năng thông qua của kênh là:
\[
\Rightarrow C' = F \log_2\left(1 + \frac{S}{N}\right) = 33888{,}6\ \text{(bit/s)}
\]

b) SNR tối thiểu để kênh có thể truyền tín hiệu thoại số có tốc độ 4800bps:
\[
\begin{aligned}
&\Rightarrow C' \geq 4800 \\
&\Leftrightarrow W \log_2(1 + \text{SNR}) \geq 4800 \\
&\Leftrightarrow \log_2(1 + \text{SNR}) \geq \frac{4800}{3400} = \frac{24}{17} \\
&\Leftrightarrow 1 + \frac{S}{N} \geq 2^{24/17} \\
&\Rightarrow \frac{S}{N} \geq 2^{24/17} - 1 \\
&\Rightarrow \text{SNR} \geq 10 \log_{10}\left(2^{24/17} - 1\right) \approx 2.203 \ (\text{dB})
\end{aligned}
\]

\newpage
\textbf{Câu 5 (3.22):} Cho sơ đồ kênh rời rạc không nhớ(DMC) như hình vẽ. Biết thời hạn phát các ký hiệu phát 0 và 1 đều là $T_p$.

a) Hãy tính dung lượng của kênh 

b) Trong trường hợp kênh nhị phân đối xứng ($\alpha=0$) dung lượng kênh bằng bao nhiêu

\[
\begin{tikzpicture}[>=latex, thick]

% Các nút trái (đầu vào)
\node[left] (x0) at (0,3) {$0$};
\node[left] (x1) at (0,0) {$1$};

% Các nút phải (đầu ra)
\node[right] (y0) at (6,3) {$0$};
\node[right] (y1) at (6,0) {$1$};
\node[right] (ye) at (6,1.5) {$e$};

% Các mũi tên từ 0
\draw[->] (x0) -- (y0) node[midway, above] {$1 - \alpha - \epsilon$};
\draw[->] (x0) -- (y1) node[midway, sloped, above] {$\epsilon$};
\draw[->] (x0) -- (ye) node[midway, sloped, below] {$\alpha$};

% Các mũi tên từ 1
\draw[->] (x1) -- (y1) node[midway, below] {$1 - \alpha - \epsilon$};
\draw[->] (x1) -- (y0) node[midway, sloped, below] {$\epsilon$};
\draw[->] (x1) -- (ye) node[midway, sloped, above] {$\alpha$};
\end{tikzpicture}
\]

\[
\textbf{Bài làm}
\]

a)\[
C = \max_{P_X} I(X, Y) = \max_{P_X} \left\{ H(Y) - H(Y \mid X) \right\}
\]

Theo định nghĩa:

\[
H(Y \mid X) = -\sum_{j=1}^{2} \sum_{k=1}^{3} p(x_j, y_k) \log p(y_k \mid x_j)
\]

Giả sử $p(x_1)=p$ $\Rightarrow$ $p(x_2)=1-p$, thay vào công thức trên ta có:


\[
H(Y \mid X) = -\sum_{j=1}^{2} \sum_{k=1}^{3} p(x_j, y_k) \log p(y_k \mid x_j)
\]

$= - 
p(x_1, y_1) \log p(y_1 \mid x_1) -
p(x_1, y_2) \log p(y_2 \mid x_1) -
p(x_1, y_3) \log p(y_3 \mid x_1) -
p(x_2, y_1) \log p(y_1 \mid x_2) -
p(x_2, y_2) \log p(y_2 \mid x_2) -
p(x_2, y_3) \log p(y_3 \mid x_2)
$

$= - 
p(x_1) p(y_1 \mid x_1) \log p(y_1 \mid x_1) -
p(x_1) p(y_2 \mid x_1) \log p(y_2 \mid x_1) -
p(x_1) p(y_3 \mid x_1) \log p(y_3 \mid x_1) -
p(x_2) p(y_1 \mid x_2) \log p(y_1 \mid x_2) -
p(x_2) p(y_2 \mid x_2) \log p(y_2 \mid x_2) -
p(x_2) p(y_3 \mid x_2) \log p(y_3 \mid x_2)
$

$= -
p(1 - \alpha - \epsilon)\log(1 - \alpha - \epsilon) -
(1-p)\epsilon\log(\epsilon) -
p\alpha\log(\alpha) -
(1-p)(1 - \alpha - \epsilon)\log(1 - \alpha - \epsilon) -
p\epsilon\log(\epsilon) -
(1-p)\alpha\log(\alpha)$

$= -
(1 - \alpha - \epsilon)\log(1 - \alpha - \epsilon) -
\epsilon\log(\epsilon) -
\alpha\log(\alpha)
$

$\Rightarrow \text{Không xuất hiện p} \Rightarrow \text{không phụ thuộc vào } X$

$\Rightarrow C_(max) \Rightarrow H(Y)_(max)$

Ta có:
\[
\begin{aligned}
p(y_1) &= p(x_1, y_1) + p(x_2, y_1) \\
       &= p(x_1) \cdot p(y_1 \mid x_1) + p(x_2) \cdot p(y_1 \mid x_2) \\
       &= p \cdot (1 - \alpha - \epsilon) + (1 - p) \cdot \epsilon \\
\\
p(y_2) &= p(x_1, y_2) + p(x_2, y_2) \\
       &= p(x_1) \cdot p(y_2 \mid x_1) + p(x_2) \cdot p(y_2 \mid x_2) \\
       &= p \cdot \epsilon + (1 - p) \cdot (1 - \alpha - \epsilon) \\
\\
p(y_3) &= p(x_1, y_e) + p(x_2, y_e) \\
       &= p(x_1) \cdot p(y_e \mid x_1) + p(x_2) \cdot p(y_e \mid x_2) \\
       &= p \cdot \alpha + (1 - p) \cdot \alpha \\
       &= \alpha
\end{aligned}
\]

$\Rightarrow p(y_1)+p(y_2)=1-\alpha$

\[
\begin{aligned}
H(Y) &= -\sum_{k=1}^{3} p(y_k) \log p(y_k) \\
     &= - \Big[ 
         p(y_1) \log p(y_1)
       + p(y_2) \log p(y_2)
       + p(y_3) \log p(y_3)
     \Big] \\
     &= - \Big[
      p(y_1)\log(p(y_1)+p(y_2)\log(p_y2)+  
      \alpha \log \alpha
     \Big]
\end{aligned}
\]

$\Rightarrow H(Y)_(max) \Leftrightarrow -p(y_1)\log(p(y_1)-p(y_2)\log(p(y_2)_(max)$

$H_2(Y) \Leftrightarrow -p(y_1)\log(p(y_1)-p(y_2)\log(p(y_2)$

$\Rightarrow H_2(Y)_(max) \Leftrightarrow p(y_1)=p(y_2)=\frac{1-\alpha}{2} $

\[
\begin{aligned}
&\Rightarrow p(1 - \alpha - \epsilon) + (1 - p)\epsilon = \frac{1 - \alpha}{2} \\
&\Rightarrow p=\frac{1}{2}
\end{aligned}
\]

\[
\begin{aligned}
\Rightarrow H_2(Y) &= -\frac{1 - \alpha}{2} \cdot \log_2\left(\frac{1 - \alpha}{2}\right).2 \\
\Rightarrow H(Y) &= (1 - \alpha) - (1 - \alpha) \log_2(1 - \alpha) - \alpha \log_2 \alpha
\end{aligned}
\]

\text{Vậy dung lượng của kênh là:}

\[
C = (1 - \alpha) - (1 - \alpha) \log_2(1 - \alpha) - (1 - \alpha - \epsilon) \log_2(1 - \alpha - \epsilon) + \epsilon \log_2 \epsilon
\]


b) Khi $\alpha=0$ ta có dung lượng kênh bằng:

\[
C=1+(1-\epsilon)\log(1-\epsilon)+\epsilon\log(\epsilon)
\]

\newpage
\textbf{Câu 6 (1.15)}: Phát biểu định lý mã hóa thứ nhất của Shannon.

\[
\textbf{Bài làm}
\]

Luôn luôn có thể xây dựng được một phép mã hóa có tín hiệu rời rạc có hiệu quả mà $\bar{n}$ có thể nhỏ tùy ý nhưng không thể nhỏ hơn Entropy của nguồn A được xác định bởi đặc tính thống kê của nguồn 

\[
\bar{n} \geq H(A)
\]

\newpage
\textbf{Câu 7 (3.12):} Hãy thực hiện mã hoá Huffman cho nguồn rời rạc sau: 

\[
A = \left(
\begin{array}{cccccccc}
a_1 & a_2 & a_3 & a_4 & a_5 & a_6 & a_7 & a_8 \\
0{,}25 & 0{,}20 & 0{,}15 & 0{,}12 & 0{,}10 & 0{,}05 & 0{,}08 & 0{,}05
\end{array}
\right)
\]

Đánh giá hiệu quả của phép mã hoá.\\

Hãy thực hiện giải mã cho dãy bit nhận được có dạng: 
\texttt{11001110101000111…}

\[
\textbf{Bài làm}
\]

\begin{figure}[h]
    \centering
    \includegraphics[width=1\linewidth]{LTTT1.png}
    \label{fig:enter-label}
\end{figure}

\textit{Mã hoá Huffman:}

\begin{center}
\begin{tabular}{|c|c|}
\hline
Ký hiệu & Mã Huffman \\
\hline
$a_1$ & 01 \\
$a_2$ & 11 \\
$a_3$ & 001 \\
$a_4$ & 101 \\
$a_5$ & 100 \\
$a_6$ & 00000 \\
$a_7$ & 0001 \\
$a_8$ & 00001 \\
\hline
\end{tabular}
\end{center}

\[
\bar{n} = 2 \cdot 0{,}25 + 2 \cdot 0{,}2 + 3 \cdot 0{,}15 + 3 \cdot 0{,}12 + 3 \cdot 0{,}1 + 5 \cdot 0{,}05 + 4 \cdot 0{,}08 + 5 \cdot 0{,}05 = 2{,}83\ \text{(dấu)}
\]

\[
\begin{aligned}
H(A) &= 0{,}25 \cdot \log_2 \frac{1}{0{,}25} + 0{,}2 \cdot \log_2 \frac{1}{0{,}2} + 0{,}15 \cdot \log_2 \frac{1}{0{,}15} \\
&\quad + 0{,}12 \cdot \log_2 \frac{1}{0{,}12} + 0{,}10 \cdot \log_2 \frac{1}{0{,}10} + 0{,}08 \cdot \log_2 \frac{1}{0{,}08} \\
&\quad + 0{,}05 \cdot \log_2 \frac{1}{0{,}05} + 0{,}05 \cdot \log_2 \frac{1}{0{,}05} \approx 2{,}798\ \text{(bit)}
\end{aligned}
\]

\[
\Rightarrow \frac{H(A)}{\bar{n}} \approx 0{,}99 \Rightarrow \text{Mã hoá gần tối ưu}
\]

\textit{Giải mã:}

\[
\texttt{11} \quad \texttt{001} \quad \texttt{11} \quad \texttt{01} \quad \texttt{01} \quad \texttt{0001} \quad \texttt{11} \ldots
\]

\[
\Rightarrow a_2,\ a_3,\ a_2,\ a_1,\ a_1,\ a_7,\ a_2,\ \ldots
\]

\newpage
\textbf{Câu 8 (1.24):} Cho kênh nhị phân đối xứng BSC với xác suất lỗi bit $p_e = 0{,}01$.


a) Tính xác suất nhận được $m$ bit sai trong $n$ bit truyền đi ($m < n$).

b) Tính xác suất nhận được chuỗi 15 bit trong đó có ít hơn 3 bit sai.


\[
\textbf{Bài làm}
\]

\textbf{a)} p_{\text{sai}} = C_n^m \cdot p_e^m \cdot (1 - p_e)^{n - m}



\textbf{b)} p = p_2 + p_1 + p_0 = C_{15}^2 \cdot p_e^2 \cdot (1 - p_e)^{13} + C_{15}^1 \cdot p_e \cdot (1 - p_e)^{14} + C_{15}^0 \cdot (1 - p_e)^{15}

\Rightarrow p \approx 0{,}988

\newpage
\textbf{Câu 9 (3.3):} Cho mã khối tuyến tính (7,3) với ma trận sinh:

\[
G_{3 \times 7} = 
\begin{pmatrix}
1 & 0 & 0 & 1 & 1 & 1 & 0 \\
0 & 1 & 0 & 0 & 1 & 1 & 1 \\
0 & 0 & 1 & 1 & 0 & 1 & 1 \\
\end{pmatrix}
\]

\begin{itemize}
    \item[a)] Tìm ma trận kiểm tra H cho bộ mã
    \item[b)] Tìm khoảng cách Hamming của bộ mã
    \item[c)] Cho bản tin đầu vào $m = 110$, tìm từ mã tương ứng.
\end{itemize}

\[
\textbf{Bài làm}
\]

a)

\[
G_{3 \times 7} = 
\begin{bmatrix}
1 & 0 & 0 & 1 & 1 & 1 & 0 \\
0 & 1 & 0 & 0 & 1 & 1 & 1 \\
0 & 0 & 1 & 1 & 0 & 1 & 1 \\
\end{bmatrix} = [I_3 \; | \; P]
\]

\[
\Rightarrow H_{4 \times 7} = [P^T \; | \; I_4] = 
\begin{bmatrix}
1 & 0 & 1 & 1 & 0 & 0 & 0 \\
1 & 1 & 0 & 0 & 1 & 0 & 0 \\
1 & 1 & 1 & 0 & 0 & 1 & 0 \\
0 & 1 & 1 & 0 & 0 & 0 & 1 \\
\end{bmatrix}
\]

b) Có cột 1, 4, 5, 6 phụ thuộc tuyến tính $\rightarrow d_0 = 4$

c) $m = 110$

Ta có: $c = mG$

\[
\rightarrow (c_1; c_2; c_3; c_4; c_5; c_6; c_7) = (1; 1; 0)
\cdot 
\begin{bmatrix}
1 & 0 & 0 & 1 & 1 & 1 & 0 \\
0 & 1 & 0 & 0 & 1 & 1 & 1 \\
0 & 0 & 1 & 1 & 0 & 1 & 1 \\
\end{bmatrix}
\]

\[
\rightarrow c_1 = 1.1 + 1.0 + 0.0 = 1; \quad c_2 = 1; \quad c_3 = 0; \quad c_4 = 1; \quad c_5 = 0; \quad c_6 = 0; \quad c_7 = 1
\]

\[
\Rightarrow (c_1; c_2; c_3; c_4; c_5; c_6; c_7) = (1; 1; 0; 1; 0; 0; 1)
\]

\newpage
\textbf{Câu 10 (2.26)}: Phân tích nhị thức $X^7 + 1$  thành tích các đa thức bất khả quy và mô tả tất cả các mã cyclic có độ dài  $n = 7$ trên vành đa thức  $\text{O\kern-0.4em/\kern0.1em}_2[X]/(X^7 + 1)$

\[
\textbf{Bài làm}
\]

Ta có đa thức bất khả quy:
\[
x^7 + 1 = (x + 1)(x^3 + x + 1)(x^3 + x^2 + 1)
\]
Các mã cyclic có độ dài n=7 là:
\begin{table}[h!]
\centering
\begin{tabular}{|c|c|c|c|}
\hline
\textbf{STT} & \boldmath$g(x)$ & \textbf{Mã (n, k)} & \boldmath$d_0$ \\
\hline
1 & $1 + x$ & (7,6) & 2 \\
2 & $1 + x + x^3$ & (7,4) & 3 \\
3 & $1 + x^2 + x^3$ & (7,4) & 3 \\
4 & $(1 + x)(1 + x + x^3)$ & (7,3) & 4 \\
5 & $(1 + x)(1 + x^2 + x^3)$ & (7,3) & 4 \\
6 & $(1 + x + x^2)(1 + x + x^3)$ & (7,1) & 7 \\
7 & $1$ & (7,7) & 1 \\
\hline
\end{tabular}
\caption{Các bộ mã cyclic với $n = 7$ trên $\mathbb{F}_2[x]/(x^7 + 1)$}
\end{table}

\newpage
\textbf{Câu 11 (4.2)}: Cho mã cyclic (7,3) với đa thức sinh \( g(x) = 1 + x^2 + x^3 + x^4 \).

\begin{enumerate}
    \item[a.] Vẽ sơ đồ mã hóa theo phương pháp chia.
    \item[b.] Khoảng cách Hamming của bộ mã bằng bao nhiêu?
    \item[c.] Vẽ sơ đồ giải mã cho mã này theo phương pháp tổng kiểm tra trực giao.
    \item[d.] Giả sử phía thu nhận được từ mã \( c = x^2 + x^4 + x^6 = 0010101 \). Thực hiện giải mã để tìm ra từ mã đã phát.
\end{enumerate}

\[
\textbf{Bài làm:}
\]

\textbf{a) Mã cyclic (7,3)}

\[
g(x) = 1 + x^2 + x^3 + x^4
\]
\[
r = 4;\quad g_0 = g_2 = g_3 = g_4 = 1;\quad g_1 = 0
\]

Sơ đồ mã hóa:

\text{b) Ma trận sinh $G$}

\[
G = 
\begin{bmatrix}
g(x) \\
xg(x) \\
x^2g(x)
\end{bmatrix}
=
\begin{bmatrix}
1 & 0 & 1 & 1 & 1 & 0 & 0 \\
0 & 1 & 0 & 1 & 1 & 1 & 0 \\
0 & 0 & 1 & 0 & 1 & 1 & 1
\end{bmatrix}
\]

\[
h(x) = \frac{x^7 + 1}{g(x)} = \frac{x^7 + 1}{x^4 + x^3 + x^2 + 1} = x^3 + x^2 + 1 \Rightarrow h^*(x) = 1 + x + x^3
\]

\[
H(x) = 
\begin{bmatrix}
h^*(x) \\
x h^*(x) \\
x^2 h^*(x) \\
x^3 h^*(x)
\end{bmatrix}
=
\begin{bmatrix}
1 & 1 & 0 & 1 & 0 & 0 & 0 \\
0 & 1 & 1 & 0 & 1 & 0 & 0 \\
0 & 0 & 1 & 1 & 0 & 1 & 0 \\
0 & 0 & 0 & 1 & 1 & 0 & 1
\end{bmatrix}
\]

Cột $1, 4, 6, 7$ (phụ thuộc tuyến tính) $\Rightarrow d_o = 4$

\text{c) Giải mã}

\[
\text{Gọi dạng của từ mã phía thu nhận được là}: c_0, c_1, c_2, c_3, c_4, c_5, c_6
\]

\text{Ta có:} c . $H^T$ = $(c_0 c_1 c_2 c_3 c_4 c_5 c_6)$ . $\begin{bmatrix}
1 & 0 & 0 & 0 \\
1 & 1 & 0 & 0 \\
0 & 1 & 1 & 0 \\
1 & 0 & 1 & 1 \\
0 & 1 & 0 & 1 \\
0 & 0 & 1 & 0 \\
0 & 0 & 0 & 1
\end{bmatrix}$ = s

\[
\Rightarrow s = (c_0 + c_1 + c_3, c_1 + c_2 + c_4, c_2 + c_3 + c_5, c_3 + c_4 + c_6)
\]
Chọn hệ tổng kiểm trực giao với $c_3$:
$\begin{cases}
s_0 = c_0 + c_1 + c_3 \\
s_1 = c_2 + c_3 + c_5 \\
s_2 = c_3 + c_4 + c_6
\end{cases}$

Ta có hệ tổng kiểm trực giao với $c_6$:
$\begin{cases}
s_0 = c_3 + c_4 + c_3 \\
s_1 = c_5 + c_6 + c_1 \\
s_2 = c_6 + c_0 + c_2
\end{cases}$

Giải mã từ mã nhận được: $c(x) = x^2 + x^4 + x^5 + x^6 \Rightarrow 0010101$

\begin{table}[h]
\centering
\begin{tabular}{|c|c c c c c c c|c c c|c|c|}
\hline
\textbf{Nhịp} & $C_0$ & $C_1$ & $C_2$ & $C_3$ & $C_4$ & $C_5$ & $C_6$ & $S_0$ & $S_1$ & $S_2$ & E & Ra \\ 
\hline
7  & 0 & 0 & 1 & 0 & 1 & 0 & 1 &   &   &   &   &   \\
8  & 1 & 0 & 0 & 1 & 0 & 1 & 0 & 0 & 1 & 0 & 0 & 1  \\
9  & 0 & 1 & 0 & 0 & 1 & 0 & 1 & 1 & 1 & 1 & 1 & 1  \\
10 & 1 & 0 & 1 & 0 & 0 & 1 & 0 & 0 & 0 & 1 & 0 & 1  \\
11 & 0 & 1 & 0 & 1 & 0 & 0 & 1 & 0 & 1 & 0 & 0 & 0  \\
12 & 1 & 0 & 1 & 0 & 1 & 0 & 0 & 0 & 0 & 1 & 0 & 1  \\
13 & 0 & 1 & 0 & 1 & 0 & 1 & 0 & 1 & 0 & 0 & 0 & 0  \\
14 & 0 & 0 & 1 & 0 & 1 & 0 & 1 & 1 & 0 & 0 & 0 & 0  \\
\hline
\end{tabular}
\caption{Bảng kiểm tra và sửa lỗi mã hóa}
\end{table}
\[
\Rightarrow c(x) = x^2 + x^4 + x^5 + x^6 \Rightarrow 0010111
\]
Vị trí $x^5$ sai đã được sửa.

Kiểm tra lại:

\newpage
\textbf{Câu 1}: Nêu định nghĩa và tính chất của lượng thông tin chéo.

\[\textbf{Giải}\]

   \textbf{Định nghĩa}: Lượng thông tin chéo là lượng thông tin trung bình truyền được qua kênh khi thực hiện phát và thu một tin bất kì.  

\[
I(A, B) = \sum_{i=1}^{s} \sum_{j=1}^{t} p(a_i, b_j) \log \frac{p(a_i \mid b_j)}{p(a_i)}
\]

\textbf{Các tính chất}

$
0 \leq I(A; B) \leq H(A), \quad 0 \leq I(A; B) \leq H(B)
$

\begin{itemize}
    \item Xảy ra đẳng thức bên phải khi và chỉ khi \( A \) và \( B \) độc lập.
    \item Xảy ra đẳng thức bên trái khi và chỉ khi kênh lý tưởng không nhiễu.
\end{itemize}

$
I(A; B) = I(B; A)
$

\begin{itemize}
    \item Lượng thông tin mà \( A \) cho biết về \( B \) cũng bằng lượng thông tin mà \( B \) cho biết về \( A \).
\end{itemize}

$
I(A; B) = H(A) - H(A|B) = H(B) - H(B|A)
$

\begin{itemize}
    \item \( I(A; B) \): lượng giảm độ bất định trung bình của \( A \) do việc biết \( B \).
\end{itemize}

$
I(A; B) = H(A) + H(B) - H(A, B)
$

$
I(A; A) = H(A)
$

\begin{itemize}
    \item \( H(A) \): lượng thông tin riêng trung bình của \( A \).
\end{itemize}







\newpage
\textbf{Câu 2}: Trong một bộ tú lơ khơ 52 quân bài (không kể phăng teo), A rút ra 1 quân bài bất kì. Tính số câu hỏi trung bình tối thiểu mà B cần đặt ra cho A để xác định được quân bài mà A rút(câu hỏi có dạng trả lời có - không hoặc đúng - sai). Nêu thuật toán hỏi. Giả sử A rút được 5 cơ, hãy nêu các câu hỏi cần thiết của B, các câu trả lời tương ứng của A và phán đoán tương ứng của B

\[\textbf{Giải}\]

Xác suất rút 1 quân bài bất kì là: \[ p(x_i) = \frac{1}{52} \]

Độ bất định của quân bài là: \( I(x_i) = -\log(p(x_i)) = \log(52) \) (\text{bit})
    

Mỗi lần đo, kết quả của phép đo thuộc nguồn rời rạc:
\[
A = \begin{pmatrix}
\text{Đúng} & \text{Sai} \\
p & 1 - p
\end{pmatrix}
\]

\text{Mỗi lần hỏi lượng thông tin trung bình nhận được là:}

\[
H(A) = -p \log(p) - (1 - p) \log(1 - p) \quad (\text{bit})
\]

Số lần hỏi trung bình để triệt tiêu hết log(52) bit độ bất định là: 
\[
n = \frac{I(x)}{H(A)}
\]
\[
\Rightarrow n_{\min} \Leftrightarrow H(A)_{\max} = \log(2) = 1 \ (\text{bit})
\Rightarrow n_{\min} = \left[ \frac{\log(52)}{1} \right] = 6 \ \text{lần}
\]
Hỏi

\begin{enumerate}
    \item Màu đỏ đúng không? Đúng $\Rightarrow$ B đoán được là rô hoặc cơ.
    \item Rô đúng không? Sai $\Rightarrow$ B đoán được nằm trong các quân A, 2, 3, \ldots, 10, J, Q, K cơ.
    \item Nằm trong A, 2, 3, 4, 5, 6, 7 cơ đúng không? Đúng $\Rightarrow$ B biết nằm trong 7 quân bài.
    \item Lẻ đúng không? Đúng $\Rightarrow$ B đoán được là 1 trong các quân A, 3, 5, 7 cơ.
    \item Không phải quân A hoặc 3 cơ có phải không? Đúng $\Rightarrow$ B đoán được quân bài là 5 hoặc 7 cơ.
    \item 5 cơ đúng không? Đúng $\Rightarrow$ Biết được quân bài là 5 cơ.
\end{enumerate}








\newpage
\textbf{Câu 3}: Tín hiệu có băng tần W=3,4kHz

a. Tính khả năng thông qua của kênh với điều kiện SNR=30db

b. Tính SNR tối thiểu cần thiết để kênh có thể truyền tín hiệu thoại số có tốc độ 4800bps
\[
\textbf{Giải}
\]
a) Có W=3400(Hz)=F
\[
SNR=10 \log_{10} \left( \frac{S}{N} \right)=30 (dB)
\]
\[
\Rightarrow \frac{S}{N} = 10^3
\]
Khả năng thông qua của kênh là:
\[
\Rightarrow C' = F \log_2\left(1 + \frac{S}{N}\right) = 33888{,}6\ \text{(bit/s)}
\]

b) SNR tối thiểu để kênh có thể truyền tín hiệu thoại số có tốc độ 4800bps:
\[
\begin{aligned}
&\Rightarrow C' \geq 4800 \\
&\Leftrightarrow W \log_2(1 + \text{SNR}) \geq 4800 \\
&\Leftrightarrow \log_2(1 + \text{SNR}) \geq \frac{4800}{3400} = \frac{24}{17} \\
&\Leftrightarrow 1 + \frac{S}{N} \geq 2^{24/17} \\
&\Rightarrow \frac{S}{N} \geq 2^{24/17} - 1 \\
&\Rightarrow \text{SNR} \geq 10 \log_{10}\left(2^{24/17} - 1\right) \approx 2.203 \ (\text{dB})
\end{aligned}
\]

\newpage
\textbf{Câu 4}: Trọng số của một từ mã $w(\alpha_i^{\ n})$: Định nghĩa và tính chất
\[
\textbf{Giải}
\]
\textbf{Định nghĩa}: Trọng số của một từ mã: $w(\alpha_i^{\ n})$ là số dấu mã khác 0 trong từ mã
\textbf{Tính chất}:
\[
0 \leq w(\alpha_i^n) \leq n
\]
\[
w(\alpha_i^n + \alpha_j^n) = d(\alpha_i^n, \alpha_j^n)
\]


%% =================
%%		Nội dung bài làm sẽ kết thúc ở đây
%% =================

\newpage
\textbf{Câu 5}: Trong phần mã hóa nguồn - Nén dữ liệu, chúng ta nói rằng các bộ mã sử dụng cho nén dữ liệu thường là các bộ mã không đều. Hãy giải thích một cách rõ ràng nhất có thể về kết luận trên.
\[
\textbf{Giải}
\]
Trong nén dữ liệu, mục tiêu là giảm số bit cần thiết để biểu diễn dữ liệu. Nếu dùng \textbf{bộ mã đều} (các ký hiệu có cùng độ dài mã), ta không tận dụng được tần suất xuất hiện của các ký hiệu.

Ngược lại, \textbf{bộ mã không đều} gán mã ngắn cho ký hiệu xuất hiện nhiều, và mã dài cho ký hiệu hiếm, giúp giảm độ dài trung bình của toàn bộ chuỗi mã $\Rightarrow$ tăng hiệu quả nén.

\textbf{Ví dụ:}

Với chuỗi \texttt{AAAAAABBC} (A: 6 lần, B: 2, C: 1)

\begin{itemize}
    \item \textbf{Mã đều}: A = 00, B = 01, C = 10
    \[
    \text{Tổng số bit} = 6 \times 2 + 2 \times 2 + 1 \times 2 = 18 \text{ bit}
    \]
    
    \item \textbf{Mã không đều}: A = 0, B = 10, C = 11
    \[
    \text{Tổng số bit} = 6 \times 1 + 2 \times 2 + 1 \times 2 = 12 \text{ bit}
    \]
    \[
    \Rightarrow \text{Tiết kiệm được } 18 - 12 = 6 \text{ bit}
    \]
\end{itemize}

Để tối ưu, $\bar{n}$ nhỏ nhất:

\[
\bar{n} = \sum_{i=1}^{S} n_i \cdot p(a_i)
\]

Mà $n_i \geq -\log p(a_i)$ nên:

\[
\Rightarrow\bar{n} \geq \sum_{i=1}^{S} p(a_i) \cdot \log \frac{1}{p(a_i)} = H(A)
\]

$\Rightarrow \bar{n}_{\min} \Leftrightarrow H(A)_{\min}$.

Các từ mã có độ dài càng nhỏ sẽ được dùng cho các tín có xác suất xuất hiện càng lớn, và ngược lại.

$\bar{n}$ nhỏ khi $H(A)$ nhỏ, cần sử dụng các bộ mã không đều.

\newpage
\textbf{câu 6}: Cho một mã khối tuyến tính có ma trận sinh G dưới đây:
\[
G = \left(
\begin{array}{ccccccccccccc}
1 & 1 & 0 & 1 & 1 & 0 & 0 & 1 & 1 & 0 & 1 & 0 & 0 \\
1 & 0 & 1 & 1 & 0 & 1 & 0 & 1 & 0 & 1 & 0 & 1 & 0 \\
1 & 1 & 1 & 0 & 0 & 0 & 1 & 1 & 1 & 1 & 0 & 0 & 1
\end{array}
\right)
\]

a) Tìm ma trận kiểm tra H ?

b) Hỏi mã này có khoảng cách Hamming bằng bao nhiều ?
\[
\textbf{Giải}
\]
\[
a) G _{3\times 13} \Rightarrow n = 13,\ k = 3 \Rightarrow r = n - k = 10
\Rightarrow H _{10 \times 13}
\]
\[
\text{Ta có } G = [P \mid I_3] \Rightarrow H = [I_{10} \mid P^T]
\]
Ma trận kiểm tra H là:
\[
H = \left[
\begin{array}{cccccccccc|ccc}
1 & 0 & 0 & 0 & 0 & 0 & 0 & 0 & 0 & 0 & 1 & 1 & 1 \\
0 & 1 & 0 & 0 & 0 & 0 & 0 & 0 & 0 & 0 & 1 & 0 & 1 \\
0 & 0 & 1 & 0 & 0 & 0 & 0 & 0 & 0 & 0 & 0 & 1 & 1 \\
0 & 0 & 0 & 1 & 0 & 0 & 0 & 0 & 0 & 0 & 1 & 1 & 0 \\
0 & 0 & 0 & 0 & 1 & 0 & 0 & 0 & 0 & 0 & 1 & 0 & 0 \\
0 & 0 & 0 & 0 & 0 & 1 & 0 & 0 & 0 & 0 & 0 & 1 & 0 \\
0 & 0 & 0 & 0 & 0 & 0 & 1 & 0 & 0 & 0 & 0 & 0 & 1 \\
0 & 0 & 0 & 0 & 0 & 0 & 0 & 1 & 0 & 0 & 1 & 1 & 1 \\
0 & 0 & 0 & 0 & 0 & 0 & 0 & 0 & 1 & 0 & 1 & 0 & 1 \\
0 & 0 & 0 & 0 & 0 & 0 & 0 & 0 & 0 & 1 & 0 & 1 & 1 \\
\end{array}
\right]
\]
b)
\[
\text{Có } c_1 + c_2 + c_4 + c_5 + c_8 + c_9 + c_{11} = 0 
\Rightarrow \text{Các cột } 1, 2, 4, 5, 8, 9, 11 \text{ phụ thuộc tuyến tính} 
\Rightarrow d_0 = 7
\]
\[
\Rightarrow
\text{Khoảng cách Hamming: $d_0=7$}
\]

\newpage
\textbf{Câu 7}:Phân tích nhị thức $x^7 + 1$  thành tích các đa thức bất khả quy và mô tả tất cả các mã cyclic có độ dài  $n = 7$ trên vành đa thức  $\text{O\kern-0.4em/\kern0.1em}_2[x]/(x^7 + 1)$
\[
\textbf{Giải}
\]
Ta có đa thức bất khả quy:
\[
x^7 + 1 = (x + 1)(x^3 + x + 1)(x^3 + x^2 + 1)
\]
Các mã cyclic có độ dài n=7 là:
\begin{table}[h!]
\centering
\begin{tabular}{|c|c|c|c|}
\hline
\textbf{STT} & \boldmath$g(x)$ & \textbf{Mã (n, k)} & \boldmath$d_0$ \\
\hline
1 & $1 + x$ & (7,6) & 2 \\
2 & $1 + x + x^3$ & (7,4) & 3 \\
3 & $1 + x^2 + x^3$ & (7,4) & 3 \\
4 & $(1 + x)(1 + x + x^3)$ & (7,3) & 4 \\
5 & $(1 + x)(1 + x^2 + x^3)$ & (7,3) & 4 \\
6 & $(1 + x + x^2)(1 + x + x^3)$ & (7,1) & 7 \\
7 & $1$ & (7,7) & 1 \\
\hline
\end{tabular}
\caption{Các bộ mã cyclic với $n = 7$ trên $\mathbb{F}_2[x]/(x^7 + 1)$}
\end{table}

\newpage
\textbf{Câu 8}:Cho mã cyclic(7,3) có đa thức sinh $g(x)=1+x^2+x^3+x^4$. Hãy mô tả sơ đồ chức năng của thiết bị mã hóa hệ thống cho bộ mã này theo phương pháp nhân. Giả sử thông tin $a(x)=1+x^2$. Hãy tìm mã ở đầu ra của thiết bị và kiểm tra lại bằng thuật toán từ mã hệ thống theo phương pháp nhân.
\[
\textbf{Giải}
\]
\[
h(x) = \frac{x^7 + 1}{g(x)} = \frac{x^7 + 1}{x^4 + x^3 + x^2 + 1}
\]

\[
  \begin{array}{r|r}
    \dropsign{-} x^7 + 0x^6 + 0x^5 + 0x^4 + 0x^3 + 0x^2 + 0x + 1 & x^4 + x^3 + x^2 + 1 \\ \cline{2-2}
    x^7 + x^6 + x^5 + x^3 & x^3 + x^2 + 1 \\ \cline{1-1} \\[\dimexpr-\normalbaselineskip+\jot]
    \dropsign{-} x^6 + x^5 - x^3 + 0x^2 + 0x + 1 \\
                 x^6 + x^5 + x^4 + x^2 & \\ \cline{1-1} \\[\dimexpr-\normalbaselineskip+\jot]
                - x^4 - x^3 - x^2 + 0x + 1 \\
                x^4 + x^3 + x^2 + 1 \\ \cline{1-1} \\[\dimexpr-\normalbaselineskip+\jot]
                      0
  \end{array}
\
\]
\[
h(x) = \frac{x^7 + 1}{g(x)} = \frac{x^7 + 1}{x^4 + x^3 + x^2 + 1} = x^3 + x^2 + 1
\]
\[
\Rightarrow h_0 = h_2 = h_3 = 1,\quad h_1 = 0
\]

\[
\begin{aligned}
C_{4 - i} &= \sum_{j=0}^{2} h_j C_{7 - i - j} \\
         &= h_0 C_{7 - i} + h_1 C_{6 - i} + h_2 C_{5 - i} \\
         &= C_{7 - i} + C_{5 - i}, \quad \text{với } 1 \leq i \leq 4
\end{aligned}         
\]
\[
\begin{aligned}
\end{aligned}
\]

\begin{figure}[H] % Dùng [H] để giữ đúng vị trí
    \centering
    \includegraphics[width=1\linewidth]{Sodomahoa.png}
    \caption{Sơ đồ mã hóa}
    \label{fig:enter-label}
\end{figure}

Cho đa thức đặc trưng:
\[
a(x) = 1 + x^2 \Rightarrow a_0 = 1, \quad a_1 = 0, \quad a_2 = 1
\]

Bảng giá trị theo các xung nhịp:

\begin{center}
\begin{tabular}{|c|c|c|c|c|c|}
\hline
\textbf{Xung nhịp} & \textbf{Vào} & \textbf{$c_4$} & \textbf{$c_5$} & \textbf{$c_6$} & \textbf{Ra} \\
\hline
1 & 1 & 1 & 0 & 0 & 1 \\
2 & 0 & 0 & 1 & 0 & 0 \\
3 & 1 & 1 & 0 & 1 & 1 \\
4 & 0 & 0 & 1 & 0 & 0 \\
5 & 0 & 0 & 0 & 1 & 0 \\
6 & 0 & 1 & 0 & 0 & 1 \\
7 & 0 & 1 & 1 & 0 & 1 \\
\hline
\end{tabular}
\end{center}

Dãy mã kết quả:
\[
c = (1,\,1,\,0,\,0,\,1,\,0,\,1)
\]

Kiểm tra lại bằng thuật toán:\\
Có $c_6 = a_2 = 1$, $c_5 = a_1 = 0$, $c_4 = a_0 = 1$\\
$\Rightarrow c_3 = c_6 + c_4 = 0$\\
\phantom{$\Rightarrow$ }$c_2 = c_5 + c_3 = 0$\\
\phantom{$\Rightarrow$ }$c_1 = c_4 + c_2 = 1$\\
\phantom{$\Rightarrow$ }$c_0 = c_3 + c_1 = 1$\\
$\Rightarrow c = (1, 1, 0, 0, 1, 0, 1)$


\newpage
\textbf{Câu 9: } Cho sơ đồ kênh rời rạc không nhớ(DMC) như hình vẽ. Biết thời hạn phát các ký hiệu phát 0 và 1 đều là $T_p$.

a) Hãy tính dung lượng của kênh 

b) Trong trường hợp kênh nhị phân đối xứng ($\alpha=0$) dung lượng kênh bằng bao nhiêu

\[
\begin{tikzpicture}[>=latex, thick]

% Các nút trái (đầu vào)
\node[left] (x0) at (0,3) {$0$};
\node[left] (x1) at (0,0) {$1$};

% Các nút phải (đầu ra)
\node[right] (y0) at (6,3) {$0$};
\node[right] (y1) at (6,0) {$1$};
\node[right] (ye) at (6,1.5) {$e$};

% Các mũi tên từ 0
\draw[->] (x0) -- (y0) node[midway, above] {$1 - \alpha - \epsilon$};
\draw[->] (x0) -- (y1) node[midway, sloped, above] {$\epsilon$};
\draw[->] (x0) -- (ye) node[midway, sloped, below] {$\alpha$};

% Các mũi tên từ 1
\draw[->] (x1) -- (y1) node[midway, below] {$1 - \alpha - \epsilon$};
\draw[->] (x1) -- (y0) node[midway, sloped, below] {$\epsilon$};
\draw[->] (x1) -- (ye) node[midway, sloped, above] {$\alpha$};
\end{tikzpicture}
\]

\[
\textbf{Giải}
\]
a)
\[
C = \max_{P_X} I(X, Y) = \max_{P_X} \left\{ H(Y) - H(Y \mid X) \right\}
\]
Theo định nghĩa:
\[
H(Y \mid X) = -\sum_{j=1}^{2} \sum_{k=1}^{3} p(x_j, y_k) \log p(y_k \mid x_j)
\]

Giả sử $p(x_1)=p$ $\Rightarrow$ $p(x_2)=1-p$, thay vào công thức trên ta có


\[
H(Y \mid X) = -\sum_{j=1}^{2} \sum_{k=1}^{3} p(x_j, y_k) \log p(y_k \mid x_j)
\]

$= - 
p(x_1, y_1) \log p(y_1 \mid x_1) -
p(x_1, y_2) \log p(y_2 \mid x_1) -
p(x_1, y_3) \log p(y_3 \mid x_1) -
p(x_2, y_1) \log p(y_1 \mid x_2) -
p(x_2, y_2) \log p(y_2 \mid x_2) -
p(x_2, y_3) \log p(y_3 \mid x_2)
$

$= - 
p(x_1) p(y_1 \mid x_1) \log p(y_1 \mid x_1) -
p(x_1) p(y_2 \mid x_1) \log p(y_2 \mid x_1) -
p(x_1) p(y_3 \mid x_1) \log p(y_3 \mid x_1) -
p(x_2) p(y_1 \mid x_2) \log p(y_1 \mid x_2) -
p(x_2) p(y_2 \mid x_2) \log p(y_2 \mid x_2) -
p(x_2) p(y_3 \mid x_2) \log p(y_3 \mid x_2)
$

$= -
p(1 - \alpha - \epsilon)\log(1 - \alpha - \epsilon) -
(1-p)\epsilon\log(\epsilon) -
p\alpha\log(\alpha) -
(1-p)(1 - \alpha - \epsilon)\log(1 - \alpha - \epsilon) -
p\epsilon\log(\epsilon) -
(1-p)\alpha\log(\alpha)$

$= -
(1 - \alpha - \epsilon)\log(1 - \alpha - \epsilon) -
\epsilon\log(\epsilon) -
\alpha\log(\alpha)
$

$\Rightarrow \text{Không xuất hiện p} \Rightarrow \text{không phụ thuộc vào } X$

$\Rightarrow C_(max) \Rightarrow H(Y)_(max)$

Ta có:
\[
\begin{aligned}
p(y_1) &= p(x_1, y_1) + p(x_2, y_1) \\
       &= p(x_1) \cdot p(y_1 \mid x_1) + p(x_2) \cdot p(y_1 \mid x_2) \\
       &= p \cdot (1 - \alpha - \epsilon) + (1 - p) \cdot \epsilon \\
\\
p(y_2) &= p(x_1, y_2) + p(x_2, y_2) \\
       &= p(x_1) \cdot p(y_2 \mid x_1) + p(x_2) \cdot p(y_2 \mid x_2) \\
       &= p \cdot \epsilon + (1 - p) \cdot (1 - \alpha - \epsilon) \\
\\
p(y_3) &= p(x_1, y_e) + p(x_2, y_e) \\
       &= p(x_1) \cdot p(y_e \mid x_1) + p(x_2) \cdot p(y_e \mid x_2) \\
       &= p \cdot \alpha + (1 - p) \cdot \alpha \\
       &= \alpha
\end{aligned}
\]
$\Rightarrow p(y_1)+p(y_2)=1-\alpha$

\[
\begin{aligned}
H(Y) &= -\sum_{k=1}^{3} p(y_k) \log p(y_k) \\
     &= - \Big[ 
         p(y_1) \log p(y_1)
       + p(y_2) \log p(y_2)
       + p(y_3) \log p(y_3)
     \Big] \\
     &= - \Big[
      p(y_1)\log(p(y_1)+p(y_2)\log(p_y2)+  
      \alpha \log \alpha
     \Big]
\end{aligned}
\]

$\Rightarrow H(Y)_(max) \Leftrightarrow -p(y_1)\log(p(y_1)-p(y_2)\log(p(y_2)_(max)$

$H_2(Y) \Leftrightarrow -p(y_1)\log(p(y_1)-p(y_2)\log(p(y_2)$

$\Rightarrow H_2(Y)_(max) \Leftrightarrow p(y_1)=p(y_2)=\frac{1-\alpha}{2} $
\[
\begin{aligned}
&\Rightarrow p(1 - \alpha - \epsilon) + (1 - p)\epsilon = \frac{1 - \alpha}{2} \\
&\Rightarrow p=\frac{1}{2}
\end{aligned}
\]
\[
\begin{aligned}
\Rightarrow H_2(Y) &= -\frac{1 - \alpha}{2} \cdot \log_2\left(\frac{1 - \alpha}{2}\right).2 \\
\Rightarrow H(Y) &= (1 - \alpha) - (1 - \alpha) \log_2(1 - \alpha) - \alpha \log_2 \alpha
\end{aligned}
\]

\text{Vậy dung lượng của kênh là:}

\[
C = (1 - \alpha) - (1 - \alpha) \log_2(1 - \alpha) - (1 - \alpha - \epsilon) \log_2(1 - \alpha - \epsilon) + \epsilon \log_2 \epsilon
\]


b)Khi $\alpha=0$ ta có dung lượng kênh bằng:
\[
C=1+(1-\epsilon)\log(1-\epsilon)+\epsilon\log(\epsilon)
\]





\newpage
\textbf{câu 10: }Cho sơ đồ kênh trong đó nguồn tín hiệu phát phát gồm $X=(x1,x2)$. Biết xác suất phát các tín hiệu $p(x_1)=p(x_2)=0,5$
\[
\begin{tikzpicture}[>=latex, thick]

% Các nút đầu vào (trái)
\node[left] (x1) at (0,3) {$x_1$};
\node[left] (x2) at (0,0) {$x_2$};

% Các nút đầu ra (phải)
\node[right] (y1) at (5,3) {$y_1$};
\node[right] (y3) at (5,1.5) {$y_3$};
\node[right] (y2) at (5,0) {$y_2$};

% Các mũi tên từ x1
\draw[->] (x1) -- (y1) node[midway, above] {$\alpha$};
\draw[->] (x1) -- (y3) node[midway, sloped, below] {$1 - \alpha$};

% Các mũi tên từ x2
\draw[->] (x2) -- (y2) node[midway, below] {$\alpha$};
\draw[->] (x2) -- (y3) node[midway, sloped, above] {$1 - \alpha$};

\end{tikzpicture}
\]

a) Hãy tính $H(X)$

b) Hãy tính $p(X=x_n,Y=y_m)$ với $n=1,2$ và $m=1,2,3$ để từ đó tính $H(X,Y)$ dưới dạng hàm số của $\alpha$

c) Hãy tính $p(Y=y_m)$, với $m=1,2,3$, để từ đó tính $H(Y)$ dưới dang hàm số của $\alpha$ 

d) Hãy tính $I(X,Y)$ dưới hàng hàm số của $\alpha$ ? Hãy xác định giá trị của $\alpha$ khi $I(X,Y)$ đạt giá trị cực đại và khi $I(X,Y)$ đạt giá trị cực tiểu? Hãy cho biết ý nghĩa trực quan của các kênh với các giá trị cực trị của $I(X,Y)$ ?
\[
\textbf{Giải}
\]
\textbf{a) Tính $H(X)$:}

\[
H(X) = - \sum_{i=1}^{2} p(x_i) \log_2 p(x_i) = -2 \cdot \frac{1}{2} \log_2 \left( \frac{1}{2} \right) = \log_2 2 = 1
\]

\textbf{b) Tính $p(x_n, y_m)$ và $H(X,Y)$:}

\[
\begin{aligned}
p(x_1, y_1) &= p(x_1) \cdot p(y_1|x_1) = \frac{1}{2} \cdot \alpha = \frac{\alpha}{2} \\
p(x_1, y_3) &= \frac{1}{2} \cdot (1 - \alpha) = \frac{1 - \alpha}{2} \\
p(x_1, y_2) &= 0 \\
p(x_2, y_2) &= \frac{1}{2} \cdot \alpha = \frac{\alpha}{2} \\
p(x_2, y_3) &= \frac{1}{2} \cdot (1 - \alpha) = \frac{1 - \alpha}{2} \\
p(x_2, y_1) &= 0
\end{aligned}
\]

Từ đó:

\[
\begin{aligned}
H(X,Y) &= -\sum_{i=1}^{2} \sum_{j=1}^{3} p(x_i, y_j) \log_2 p(x_i, y_j) \\
&= -\left( \frac{\alpha}{2} \log_2 \frac{\alpha}{2} + \frac{1 - \alpha}{2} \log_2 \frac{1 - \alpha}{2} + \frac{\alpha}{2} \log_2 \frac{\alpha}{2} + \frac{1 - \alpha}{2} \log_2 \frac{1 - \alpha}{2} \right) \\
&= -\left( \alpha \log_2 \frac{\alpha}{2} + (1 - \alpha) \log_2 \frac{1 - \alpha}{2} \right)
\end{aligned}
\]

\textbf{c) Tính $p(y_j)$ và $H(Y)$:}

\[
\begin{aligned}
p(y_1) &= p(x_1, y_1) + p(x_2, y_1) = \frac{\alpha}{2} + 0 = \frac{\alpha}{2} \\
p(y_2) &= 0 + \frac{\alpha}{2} = \frac{\alpha}{2} \\
p(y_3) &= \frac{1 - \alpha}{2} + \frac{1 - \alpha}{2} = 1 - \alpha
\end{aligned}
\]

\[
\begin{aligned}
H(Y) &= - \sum_{j=1}^{3} p(y_j) \log_2 p(y_j) \\
&= - \left( \frac{\alpha}{2} \log_2 \frac{\alpha}{2} + \frac{\alpha}{2} \log_2 \frac{\alpha}{2} + (1 - \alpha) \log_2 (1 - \alpha) \right) \\
&= - \left( \alpha \log_2 \frac{\alpha}{2} + (1 - \alpha) \log_2 (1 - \alpha) \right)
\end{aligned}
\]

\textbf{d) Tính $I(X;Y)$ và phân tích cực trị:}

\[
\begin{aligned}
I(X;Y) &= H(X) + H(Y) - H(X,Y) \\
&= 1 - \left[ \alpha \log_2 \frac{\alpha}{2} + (1 - \alpha) \log_2 (1 - \alpha) \right] - \left[ -\left( \alpha \log_2 \frac{\alpha}{2} + (1 - \alpha) \log_2 \frac{1 - \alpha}{2} \right) \right] \\
&= 1 - \alpha \log_2 \frac{\alpha}{2} - (1 - \alpha) \log_2 (1 - \alpha) + \alpha \log_2 \frac{\alpha}{2} + (1 - \alpha) \log_2 \frac{1 - \alpha}{2} \\
&= 1 + (1 - \alpha) \left( \log_2 \frac{1 - \alpha}{2} - \log_2 (1 - \alpha) \right) \\
&= 1 + (1 - \alpha) \log_2 \frac{1 - \alpha}{2(1 - \alpha)} = 1 + (1 - \alpha) \log_2 \frac{1}{2} = 1 - (1 - \alpha)
\end{aligned}
\]

Vậy:

\[
I(X;Y) = \alpha
\]

Giá trị cực đại của $I(X;Y)$ là $\alpha = 1$, khi đó tín hiệu đầu ra hoàn toàn xác định bởi đầu vào.

Giá trị cực tiểu là $\alpha = 0$, khi đó toàn bộ tín hiệu đầu ra là $y_3$ — tức kênh hoàn toàn nhiễu, không mang thông tin.

\textbf{Ý nghĩa trực quan:}  
- Khi $\alpha = 1$: kênh không có nhiễu, đầu vào được truyền chính xác đến đầu ra $\Rightarrow$ kênh lý tưởng.  
- Khi $\alpha = 0$: mọi tín hiệu đều bị biến thành $y_3$, không thể phân biệt được đầu vào nào được gửi đi $\Rightarrow$ kênh vô ích.


\newpage
\textbf{Câu 11 :}Giá trị mức xám của một khối (block) ảnh 8x8 như trong ma trận sau.

\begin{figure}[h]
    \centering
    \includegraphics[width=0.5\linewidth]{huff.png}
    \caption{}
    \label{fig:enter-label}
\end{figure}
Người ta cần thực hiện nén ảnh này. Một cách đơn giản nhất là áp dụng cách mã hóa các mức xám theo phương pháp mã hóa Huffman

a) Hãy xây dựng bộ mã biểu diễn các giá trị mức xám của ảnh theo phương pháp mã hóa Huffman

b) Đánh giá tính hiệu quả của bộ mã thu được.

c) Giả sử quét đường zig-zag theo đường đứt nét, với dãy bit nhận được như sau 0100110100111010101 ... hãy khôi phục lại các giá trị mức xám của góc ảnh ứng với dãy bit đã cho.

d) So với việc mã hóa trực tiếp các giá trị mức xám bằng mã ASCII(độ rông 1BYTE), phương pháp mã hóa Huffman tiết kiệm được bao nhiêu dung lượng.

\[
\textbf{Giải}
\]
a)

Đếm số lần xuất hiện của mỗi mức xám:

\begin{center}
\begin{tabular}{|c|c|}
\hline
\textbf{Giá trị} & \textbf{Số lần xuất hiện} \\
\hline
0 & 16 \\
2 & 16 \\
32 & 8 \\
36 & 8 \\
68 & 4 \\
72 & 4 \\
80 & 2 \\
87 & 2 \\
92 & 2 \\
101 & 1 \\
102 & 1 \\
\hline
\end{tabular}
\end{center}
\begin{figure}[h]
    \centering
    \includegraphics[width=1\linewidth]{huff2.drawio.png}
    \caption{}
    \label{fig:enter-label}
\end{figure}

\begin{center}
\textbf{Bảng mã Huffman:}

\vspace{0.3cm}
\begin{tabular}{|c|c|}
\hline
\textbf{Giá trị xám} & \textbf{Mã Huffman} \\
\hline
0   & 11 \\
2   & 10 \\
32  & 011 \\
36  & 010 \\
68  & 0011 \\
72  & 0010 \\
80  & 00011 \\
87  & 00010 \\
92  & 00001 \\
101 & 000001 \\
102 & 000000 \\
\hline
\end{tabular}
\end{center}

b)
\[
\bar{l} = \sum_{k=1}^{11} p(x_k) \cdot l_k
= p(x_1) \cdot l_1 + p(x_2) \cdot l_2 + p(x_3) \cdot l_3 + \dots + p(x_{11}) \cdot l_{11}
\]
\[
\bar{l} = \frac{1}{64} \cdot 6 + \frac{1}{64} \cdot 6 + \frac{1}{32} \cdot 5 + \ldots + \frac{1}{4} \cdot 2
\]
\[
\Rightarrow \bar{l} = \frac{93}{32}
\]
\[
H(X) = -\sum_{k=1}^{11} p(x_k) \log_2 p(x_k) = -\left( 
p(x_1) \log_2 p(x_1) +
p(x_2) \log_2 p(x_2) +
\cdots +
p(x_{11}) \log_2 p(x_{11})
\right)
\]
\[
\Rightarrow H(X)=\frac{93}{32}
\]

\[
\Rightarrow n = \frac{H(X)}{\bar{l}} = 1 \\
\Rightarrow \text{phép mã hóa tối ưu}
\]

c)
\[
\begin{array}{|c|c|c|c|c|c|c|c|}
\hline
\text{Góc ảnh} & 010 & 011 & 010 & 011 & 10 & 10 & 10 \\
\hline
\text{Mức xám} & 36 & 32 & 36 & 32 & 2 & 2 & 2 \\
\hline
\end{array}
\]
d)

Độ dài khi dung Huffman:
\[
16.2.2+8.3.2+4.4.2+2.3.5+1.6.2 = 186 \text{(bit)}
\]
Độ dài khi dùng ASCII Là: 
\[
8.64=512\text{(bit)}
\]
\[
\Rightarrow \text{Hệ số nén}=\frac{512}{186}=2,75 
\]

\textbf{Câu 1}: Nêu định nghĩa và tính chất của entropy của nguồn rời rạc A sau:
\begin{equation*}
    A = 
    \begin{pmatrix}
    a_1 & a_2 & \dots & a_s \\
    p(a_1) & p(a_2) & \dots & p(a_s)
\end{pmatrix}
\end{equation*}

\textbf{Bài làm}: 
\\
\begin{adjustwidth}{2em}{0pt}
\text{Entropy của nguồn } A \text{ là lượng thông tin trung bình chứa trong 1 tin của nguồn } A, \text{ ký hiệu là } H(A), \text{ được định nghĩa bởi công thức:}
\\\[
H(A) = -\sum_{i=1}^{s} p(a_i) \log_2 p(a_i)
\]
\\
\text{Tính chất:}
\begin{itemize}
    \item $0 \leq H(A) \leq \log_2 s$
    
    \item $H(A) = 0 \Leftrightarrow 
    \begin{cases}
        p(a_i) = 1 \\
        p(a_j) = 0 & \forall j \ne i
    \end{cases}$
    
    \item $H(A) = \log_2 s \Leftrightarrow 
    \begin{cases}
        p(a_i) = p(a_j) = \frac{1}{s}, & \forall i, j \in \{1, \ldots, s\}
    \end{cases}$
\end{itemize}
\end{adjustwidth}


\newpage
\textbf{Câu 2}: Có hai độp đựng bút chì,mỗi hộp đựng 20 bút chì. Hộp thứ nhất có 10 bút chì trắng, 5 bút chì đen và 5 bút chì đỏ. Hộp thứ 2 có 8 bút chì trắng, 8 bút chì đen, 4 bút chì đỏ. thực hiện các 2 phép thử lấy hú họa một bút chì từ mỗi hộp. Hỏi rằng phép thử nào trong thử trên có độ bất định lớn hơn.  
\\
\\
\textbf{Bài làm}:
\\
\textbf{Hộp 1:}
\begin{itemize}
  \item Trắng: \( P(T_1) = \frac{10}{20} = 0.5 \)
  \item Đen: \( P(\Đ_1) = \frac{5}{20} = 0.25 \)
  \item Đỏ: \( P(Đỏ_1) = \frac{5}{20} = 0.25 \)
\end{itemize}

\textbf{Hộp 2:}
\begin{itemize}
  \item Trắng: \( P(T_2) = \frac{8}{20} = 0.4 \)
  \item Đen: \( P(\Đ_2) = \frac{8}{20} = 0.4 \)
  \item Đỏ: \( P(Đỏ_2) = \frac{4}{20} = 0.2 \)
\end{itemize}
Công thức entropy:
\[
H(A) = -\sum_{i=1}^{s} p(a_i) \log_2 p(a_i)
\]
 \\
 \textbf{Entropy của hộp 1:}
\[
\begin{aligned}
H_1 &= -[0.5 \log_2 0.5 + 0.25 \log_2 0.25 + 0.25 \log_2 0.25] \\
    &= -[0.5(-1) + 0.25(-2) + 0.25(-2)] \\
    &= 0.5 + 0.5 + 0.5 = \text{1.5 bits}
\end{aligned}
\]

\textbf{Entropy của hộp 2:}
\[
\begin{aligned}
H_2 &= -[0.4 \log_2 0.4 + 0.2 \log_2 0.2 + 0.4 \log_2 0.4] \\
    &\approx -[0.4 \times (-1.322) + 0.2 \times (-2.322) + 0.4 \times (-1.322)] \\
    &\approx 0.529 + 0.464 + 0.529 = \text{1.522 bits}
\end{aligned}
\]
\\ Phép thử lấy bút chì từ hộp 2 có độ bất định lớn hơn.


\newpage
\textbf{Câu 3}: Cho sơ đồ kênh trong đó nguồn tín hiệu phát gồm $X=(X_1, X_2)$. Biết xác suất phát các tín hiệu $p(x_1) = p(x_2) = 0,5.$
\begin{figure}[h]
    \centering
    \includegraphics[width=0.4\textwidth]{cau3.png}
   
    \label{fig:cau3}
\end{figure}

\begin{enumerate}
    \item[a.] Hãy tính $H(X)$.
    
    \item[b.] Hãy tính $p(X = x_n,\ Y = y_m)$, với $n = 1,2$ và $m = 1,2,3$ để từ đó tính $H(X,Y)$ dưới dạng hàm số của $\alpha$.
    
    \item[c.] Hãy tính $p(Y = y_m)$, với $m = 1,2,3$ để từ đó tính $H(Y)$ dưới dạng hàm số của $\alpha$.
    
    \item[d.] Hãy tính $I(X,Y)$ dưới dạng hàm số của $\alpha$? Hãy xác định giá trị của $\alpha$ khi $I(X,Y)$ đạt giá trị cực đại và khi $I(X,Y)$ đạt giá trị cực tiểu? Hãy cho biết ý nghĩa trực quan của các kênh ứng với các giá trị cực trị của $I(X,Y)$?
\end{enumerate}


\textbf{Câu 4.3:} Cho sơ đồ kênh trong đó nguồn tín hiệu phát gồm \( X = \{x_1, x_2\} \). Biết xác suất phát các tín hiệu: \( p(x_1) = p(x_2) = 0{,}5 \).
\begin{itemize}
    \item[a)] Hãy tính \( H(X) \).
    
    \item[b)] Hãy tính \( p(X = x_n, Y = y_m) \) với \( n = 1, 2 \) và \( m = 1, 2, 3 \), từ đó tính \( H(X, Y) \) dưới dạng hàm số \( \alpha \).
    
    \item[c)] Hãy tính \( p(Y = y_m) \) với \( m = 1, 2, 3 \), từ đó tính \( H(Y) \) dưới dạng hàm số \( \alpha \).
    
    \item[d)] Hãy tính \( I(X, Y) \) dưới dạng hàm số của \( \alpha \). Hãy xác định giá trị của \( \alpha \) khi \( I(X, Y) \) đạt giá trị cực đại và khi \( I(X, Y) \) đạt giá trị cực tiểu. Hãy cho biết ý nghĩa trực quan của kênh ứng với các giá trị cực trị của \( I(X, Y) \).
\end{itemize}

\textbf{Bài làm}: 
\\
\begin{itemize}
    \item [a.] Áp dụng công thức: \[H(X) = -\sum_{i=1}^{n} p(x_i) \log_2 p(x_i)\]
    
    \[H[X] = 0,5.\log_2 0,5 + 0,5.\log_2 0,5 = \text{1 bits} \]
    \item [b. ]\\
    $p(x_1, y_1) = 0,5\alpha$\\
    $P(x_1, y_3) = 0,5.(1-\alpha)$\\
    $p(x_2, y2) = 0,5.\alpha$\\
    $p(x_2, y_3) = 0,5(1-\alpha)$\\
    Áp dụng công thức Enttropy hợp: 
    \[
    H(X, Y) = - \sum_{k=1}^{N} \sum_{l=1}^{M} p(x_k, y_l) \log \left( p(x_k, y_l) \right)
    \]
    \text {Ta có;}
     \[
    \begin{aligned}
    \Rightarrow\ &H(X,Y) = -\left[ 0{,}5\alpha \log 0{,}5\alpha\cdot 2 + 0{,}5(1 - \alpha) \log 0{,}5(1 - \alpha) \right.\cdot 2] \\
    &= -\left[ \alpha \log \frac{\alpha}{2} + (1 - \alpha) \log \frac{1 - \alpha}{2} \right]
    \end{aligned}
    \]
    \item[c.]\\
     $p(y_1) = p(y_1|x_1) = 0{,}5\alpha$ \\
    $p(y_3) = p(y_3|x_1) + p(y_3|x_2) = 0{,}5(1 - \alpha) + 0{,}5(1 - \alpha) = 1 - \alpha$ \\
    $p(y_2) = p(y_2|x_2) = 0{,}5\alpha$ \\
    \text {Tương tự, ta có:}\\
    \[
    H(Y) = -\left[ \frac{\alpha}{2} \log \frac{\alpha}{2} + (1 - \alpha) \log (1 - \alpha) + \frac{\alpha}{2} \log \frac{\alpha}{2} \right] 
    = -\left[ \alpha \log \frac{\alpha}{2} + (1 - \alpha) \log (1 - \alpha) \right]
    \]
    \item[d. ] 
    \text{Ta có:} 
    \\
    \[
    I(X, Y) = H(X) + H(Y) - H(X, Y)
    \]
    \[
    = 1 - \left[ \alpha \log \frac{\alpha}{2} + (1 - \alpha) \log (1 - \alpha) \right] + \left[ \alpha \log \frac{\alpha}{2} + (1 - \alpha) \log \frac{1 - \alpha}{2} \right]
    \]
    \[
    = 1 + (1 - \alpha) \left[ \log \frac{1 - \alpha}{2} - \log (1 - \alpha) \right]
    = \alpha
    \]
    
    \[
    \Rightarrow I(X, Y) \text{ đạt cực đại khi } \alpha = 1 \Rightarrow \text{kênh ít nhiễu}
    \]
    \[
    I(X, Y) \text{ đạt cực tiểu khi } \alpha = 0 \Rightarrow \text{kênh rất nhiễu}
    \]

    


    

    
    
\end{itemize}
    
\newpage
\textbf{Câu 4:} Nêu định nghĩa và tính chất khả năng thông qua của nguồn rời rạc.
\\
\\
\textbf{Bài làm}: 
\\
\textbf{Khả năng thông qua của kênh rời rạc} là giá trị cực đại của lượng thông tin truyền qua kênh trong một đơn vị thời gian lấy theo mọi khả năng có thể của phân bố nguồn phát.

\[
C' = \max_{p(X)} I'(X; Y) = \max_X I'(X; Y) = v_k \max_X I(X; Y) \quad \text{[bit/s]}= v_k C
\]
   \begin{flushleft}
$C = \max_X I(X; Y)$: khả năng thông qua của kênh đối với mỗi dấu.

$C$: đơn vị là [bit/lần truyền]

$C$: thường được sử dụng.
\end{flushleft}


\textbf{Tính chất:}
\begin{itemize}
    \item $C' \geq 0$, $C' = 0$ khi và chỉ khi $X$ và $Y$ hoàn toàn độc lập $\Rightarrow$ kênh bị đứt.
    \item $C' \leq v_k \log(N)$ \quad ($N$ là độ lớn của nguồn $X$)
    \item $C' \leq v_k \log(M)$ \quad ($M$ là độ lớn của nguồn $Y$)
\end{itemize}

\newpage 

\textbf{Câu 5:} \\
Bộ mã nào dưới đây có thể hoặc không thể là mã Huffman của bất kỳ một nguồn rời rạc nào? Nếu không thể thì giải thích tại sao? \\
Nếu có thể thì hãy cho ví dụ một nguồn tin tương ứng với bộ mã đó. Chú ý, mỗi câu đã liệt kê toàn bộ các từ mã (cách nhau bởi dấu phẩy) trong một bộ mã:

\begin{enumerate}
    \item[a.] 0, 10, 111, 101
    \item[b.] 00, 010, 011, 10, 110
    \item[c.] 1, 000, 001, 010, 011
\end{enumerate}
\textbf{Bài làm:}
\\
\textbf{a.} Bộ mã: $0,\ 10,\ 111,\ 110$ không phải là bộ mã $0,\ 10,\ 111,\ 101$ \\
$\Rightarrow$ Đây không phải là mã Huffman.
\begin{center}
    \begin{center}
    \includegraphics[scale=1]{cau5.png}
\end{center}

\end{center}
\textbf{b.}
\\ $010$ và $011$ có cùng tiền tố $01$.
\\ 
\text{Hai từ mã có độ dài ngắn nhất là $00$ và $01$, đều giống nhau dấu mã đầu tiên}
\\ \text { $\Rightarrow$ không thỏa mã bộ mã Huffman}
\\
$\Rightarrow$ \text {Không phải bộ mã Huffman}
\\
\textbf{c.}  Bộ mã $1, 000, 001, 010, 011$ $\Rightarrow$ \text {Là bộ mã Huffman}
\begin{center}
    \includegraphics[scale=0.7]{cau5c.png}
\end{center}
\text{Nguồn tin $A = (x_1, x_2, x_3, x_4, x_5)$ với $P$ lần lượt là $0,125; 0,125; 0,25; 0,25; 0,25$ }
\begin{flushleft}
    \includegraphics[scale=0.7]{cau5cc.png}
\end{flushleft}

    $\Rightarrow$ \text{Mã hóa Huffman}\\
    $x_1 = 000$\\
    $x_2 = 001$\\
    $x_3 = 010$\\
    $x_4 = 011$\\
    $x_5 = 1$
    
\newpage
\textbf{Câu hỏi 6:} \\

\textbf{a.} Hãy cho biết nhược điểm của mã Huffman khi sử dụng cho mục đích nén dữ liệu? \\

\textbf{b.} Cho hai bộ mã khác nhau dùng để mã hóa cho các ký tự $a, b, c, d$. Trong bảng, $p_i$ là xác suất xuất hiện của mỗi ký tự. Hỏi chiều dài trung bình để mã hóa cho một ký tự trong mỗi bộ mã là bao nhiêu?

\begin{center}
\renewcommand{\arraystretch}{1.5}
\setlength{\tabcolsep}{20pt}
\begin{tabular}{|c|c|c|c|}
\hline
\textbf{$a_i$} & \textbf{$c_1(a_i)$} & \textbf{$c_2(a_i)$} & \textbf{$p_i$} \\
\hline
a & 1000 & 0   & $\dfrac{1}{2}$ \\
b & 0100 & 10  & $\dfrac{1}{4}$ \\
c & 0010 & 110 & $\dfrac{1}{8}$ \\
d & 0001 & 111 & $\dfrac{1}{8}$ \\
\hline
\end{tabular}
\end{center}
\textbf{Bài làm}: 
\begin{itemize}
    \item[a.] Nhược điểm của mã Huffman khi sử dụng để nén dữ liệu:
    \begin{itemize}[label=.]
        \item Tập các từ mã tối ưu không là duy nhất.
        \item Bên nhận muốn giải mã thì phải có bảng mã giống bên gửi.
    \end{itemize}
    $\Rightarrow$ Khi nén các tập tin bé, hệ số nén không cao.
    
    \item[b.] Chiều dài trung bình để mã hóa cho mỗi ký tự trong mỗi bộ mã là:
    \begin{align*}
        \bar{n}_{c_1} &= n_{c_1}(a) \cdot p(a) + n_{c_1}(b) \cdot p(b) + n_{c_1}(c) \cdot p(c) + n_{c_1}(d) \cdot p(d) \\
        &= 4 \cdot \dfrac{1}{2} + 4 \cdot \dfrac{1}{4} + 4 \cdot \dfrac{1}{8} + 4 \cdot \dfrac{1}{8} = 4 \quad \text{(dấu)} \\
        \bar{n}_{c_2} &= n_{c_2}(a) \cdot p(a) + n_{c_2}(b) \cdot p(b) + n_{c_2}(c) \cdot p(c) + n_{c_2}(d) \cdot p(d) \\
        &= 1 \cdot \dfrac{1}{2} + 2 \cdot \dfrac{1}{4} + 3 \cdot \dfrac{1}{8} + 3 \cdot \dfrac{1}{8} = 1{,}75 \quad \text{(dấu)}
    \end{align*}
\end{itemize}

\newpage
\textbf{Câu hỏi 7:} Giá trị mức xám của một khối (block) ảnh $8 \times 8$ như trong ma trận sau:
\begin{center}
    \includegraphics[scale=0.5]{cau 7 .png}
\end{center}


\begin{enumerate}
\text Người ta cần thực hiện nén ảnh này. Một cách đơn giản nhất là áp dụng cách mã hóa các mức xám theo phương pháp mã hóa Huffman.
\end{enumerate}

\begin{enumerate}
    \item[a.] Hãy xây dựng bộ mã biểu diễn các giá trị mức xám của ảnh theo phương pháp mã hóa Huffman.
    \item[b.] Đánh giá tính hiệu quả của bộ mã thu được.
    \item[c.] Giả sử ảnh được quét zig-zag theo đường dẫn nét, với dãy bit nhận được như sau: 
    \texttt{0100110100111010101...} 
    Hãy khôi phục lại các giá trị mức xám của gốc ảnh theo dãy bit đã cho.
    \item[d.] So với việc mã hóa trực tiếp các giá trị mức xám bằng mã ASCII (độ rộng 1 Byte), phương pháp mã hóa Huffman tiết kiệm được bao nhiêu phần trăm dung lượng.
\end{enumerate}
\par\noindent\textbf{Bài làm:}
\begin{itemize}
    \item [a.] Từ dữ kiện đầu bài ta có:
    \[
    \begin{array}{|c|c|c|c|c|c|c|c|c|c|c|}
    \hline
    102 & 101 & 97 & 90 & 72 & 64 & 36 & 32 & 2 & 0 \\
    \hline
    \frac{1}{64} & \frac{1}{64} & \frac{1}{32} & \frac{1}{32} & \frac{1}{16} & \frac{1}{16} & \frac{1}{8} & \frac{1}{8} & \frac{1}{7} & \frac{1}{5} \\
    \hline
    \end{array}
    \]
     \begin{center}
        \includegraphics[scale=0.5]{cau6.png}
    \end{center}   

   \begin{center}
    \textbf{Mã hóa Huffman}
    \[
    \begin{array}{|c|c|c|c|c|c|c|c|c|c|c|}
    \hline
    0 & 2 & 32 & 36 & 68 & 72 & 80 & 87 & 92 & 101 & 102 \\
    \hline
    11 & 10 & 011 & 010 & 0011 & 0010 & 00011 & 00010 & 00001 & 000001 & 000000 \\
    \hline
    \end{array}
    \]
    \end{center}
    \item[b.] Độ dài trung bình của mã:
    \begin{align*}
        L  = 2 \cdot \dfrac{1}{4}\cdot 2 + 3 \cdot \dfrac{1}{8}\cdot2 + 4 \cdot \dfrac{1}{16}\cdot 2 + 5 \cdot \dfrac{1}{32}\cdot 3 + 6\cdot \dfrac{1}{64}\cdot2 =\dfrac{93}{32} \approx 2,9\text{(dấu)} \\
    \end{align*}
    \text{Entropy của nguồn:}
    \[
    \begin{aligned}
    H(X) &= -\left( 3 \cdot \frac{1}{64} \log_2 \frac{1}{64} + 3 \cdot \frac{1}{32} \log_2 \frac{1}{32} + 2 \cdot \frac{1}{16} \log_2 \frac{1}{16} + 3 \cdot \frac{1}{8} \log_2 \frac{1}{8} \right) \\
    &= 3 \cdot \frac{6}{64} + 3 \cdot \frac{5}{32} + 2 \cdot \frac{4}{16} + 3 \cdot \frac{3}{8} \\
    &= 0.28125 + 0.46875 + 0.5 + 1.125 = 2.375
    \end{aligned}
    \]
    $\Rightarrow$ \text{Hiệu quả của bộ mã thu được:}\\
    \[
    \eta = \frac{H(X)}{L} = \frac{2.375}{2.90625} \approx 0.817 \approx 81.7\%
    \]
    \item[c. ] 
   \[
    \begin{array}{|c|c|c|c|c|c|c|c|}
    \hline
    010 & 011 & 010 & 011 & 10 & 10 & 10 & 1 \\
    \hline
    36 & 32 & 36 & 32 & 2 & 2 & 2 & \cdots \\
    \hline
    \end{array}
    \]
    \item[d.] Độ dài khi dùng mã Hufman:\\
    $16\cdot2 \cdot2 + 8\cdot 3 \cdot2 + 4\cdot4\cdot2 + 2\cdot3\cdot5+ 1\cdot6\cdot2 = 168 \text{(bit)}$\\
    \text{Độ dài khi dùng mã ASCII: $8\cdot64 = 512 \text{ (bit)}$}
    \[
    \text{Phần trăm tiết kiệm} = \left( \frac{512 - 168}{512} \right) \times 100\% = \left( \frac{344}{512} \right) \times 100\% \approx 67.19\%
    \]

\end{itemize}



\newpage
\textbf{Câu hỏi 8:} Cho hai kênh BSC được mắc nối tiếp như hình: 

\includegraphics{cau8.png}


    \text{Xác suất lỗi bit khi truyền trên kênh BSC1 và BSC2 tương ứng là $p_{e1}$ và $p_{e2}$.}\\
    \text{Tính xác suất lỗi bit khi truyền qua hai kênh này.}

\par\noindent\textbf{Bài làm:}



\newpage

\textbf{Câu hỏi 9:} 
Cho mã khối tuyến tính $(6,3)$ với ma trận sinh:
\[
G_{3 \times 6} =
\begin{pmatrix}
1 & 0 & 0 & 1 & 1 & 1 \\
0 & 1 & 0 & 0 & 1 & 1 \\
0 & 0 & 1 & 1 & 0 & 1 \\
\end{pmatrix}
\]

\begin{itemize}
  \item[a.] Tìm ma trận kiểm tra $H$ cho bộ mã.
  \item[b.] Tìm khoảng cách Hamming của bộ mã.
\end{itemize}
\par\noindent\textbf{Bài làm:}
\begin{itemize}
    \item [a.] Giả sử $G = [I_3 \mid P]$, với $I_3$ là ma trận đơn vị $3 \times 3$, và $P$ là phần còn lại $3 \times 3$:

    \[
    P = \begin{pmatrix}
    1 & 1 & 1 \\
    0 & 1 & 1 \\
    0 & 1 & 1 \\
    \end{pmatrix}
    \]
    
    Ma trận kiểm tra $H$ sẽ có dạng:    $H = \left( -P^T \mid I_3 \right)$
    
    Tính $-P^T$ trong trường nhị phân (mod 2), thì $-1 = 1$, nên:
    
    \[
    P^T = \begin{pmatrix}
    1 & 0 & 0 \\
    1 & 1 & 1 \\
    1 & 1 & 1 \\
    \end{pmatrix}
    \Rightarrow -P^T = P^T
    \]
    
    \text{Vậy:}
    $  H = \begin{pmatrix}
    1 & 0 & 0 & 1 & 0 & 0 \\
    1 & 1 & 1 & 0 & 1 & 0 \\
    1 & 1 & 1 & 0 & 0 & 1 \\
    \end{pmatrix}$
    
    \item [b.]
    \text{Liệt kê toàn bộ 7 thông điệp đầu vào \( \mathbf{u} \in \mathbb{F}_2^3 \setminus \{(0,0,0)\} \) và tính mã hóa \( \mathbf{c} = \mathbf{u} \cdot G \)}

    \[
    \begin{array}{c|c|c}
    \mathbf{u} & \mathbf{c} = \mathbf{u} \cdot G & \text{Trọng số Hamming} \\
    \hline
    (1\ 0\ 0) & (1\ 0\ 0\ 1\ 1\ 1) & 4 \\
    (0\ 1\ 0) & (0\ 1\ 0\ 0\ 1\ 1) & 3 \\
    (0\ 0\ 1) & (0\ 0\ 1\ 0\ 1\ 1) & 3 \\
    (1\ 1\ 0) & (1\ 1\ 0\ 1\ 0\ 0) & 3 \\
    (1\ 0\ 1) & (1\ 0\ 1\ 1\ 0\ 0) & 3 \\
    (0\ 1\ 1) & (0\ 1\ 1\ 0\ 0\ 0) & \textbf{2} \\
    (1\ 1\ 1) & (1\ 1\ 1\ 1\ 1\ 1) & 6 \\
    \end{array}
    \]
    $\Rightarrow$ \text{Khoảng cách hamming của bộ mã là: } $d_\text{min} = 2$

\end{itemize}


\newpage
\textbf{Câu hỏi 10:}Cho mã cyclic $(7,4)$ có đa thức sinh   $g(x) = 1 + x +  x^3 $ 
Hãy xây dựng ma trận sinh $G$ và ma trận kiểm tra $H$ ở dạng hệ thống của mã này.\\
\textbf{Bài làm:} Mã cyclic\quad $(7, 4)$, $g(x) = 1 + x + x^3$.

Ta có: 
\[
G = \begin{bmatrix}
g(x) \\
x g(x) \\
x^2 g(x) \\
x^3 g(x)
\end{bmatrix}
= 
\begin{bmatrix}
1 + x + x^3 \\
x + x^2 + x^4 \\
x^2 + x^3 + x^5 \\
x^3 + x^4 + x^6
\end{bmatrix}
\Rightarrow
\begin{bmatrix}
1 & 1 & 0 & 1 & 0 & 0 & 0 \\
0 & 1 & 1 & 0 & 1 & 0 & 0 \\
0 & 0 & 1 & 1 & 0 & 1 & 0 \\
0 & 0 & 0 & 1 & 1 & 0 & 1
\end{bmatrix}
\]

Đa thức kiểm tra:
\[
h(x) = \frac{x^7 + 1}{g(x)} = \frac{x^7 + 1}{1 + x + x^3}
\]

\[
\Rightarrow h^*(x) = 1 + x^2 + x^4 + x^5 + x^6
\]

Ta có: 
\[
H = \begin{bmatrix}
h^*(x) \\
x h^*(x) \\
x^2 h^*(x)
\end{bmatrix}
= 
\begin{bmatrix}
1 + x^2 + x^4 + x^5 + x^6 \\
x + x^3 + x^5 + x^6 + x^7 \equiv x + x^3 + x^5 + x^6 + 1 \ (\text{mod } x^7 + 1) \\
x^2 + x^4 + x^6 + x^7 + x^8 \equiv x^2 + x^4 + x^6 + 1 + x \ (\text{mod } x^7 + 1)
\end{bmatrix}

    $\Rightarrow$
    \begin{bmatrix}
    1 & 0 & 1 & 0 & 1 & 1 & 1 \\
    0 & 1 & 0 & 1 & 0 & 1 & 1 \\
    1 & 1 & 1 & 0 & 1 & 0 & 1
    \end{bmatrix}


\]

\medskip

Với $(7,4)$, $g(x) = 1 + x + x^3$

Ta có: $l = \1,2,3,4$, $\Rightarrow x^\text{n-l} = x^6, x^5, x^4, x^3$, \text{có:}

\begin{align*}
x^6 \bmod g(x) &= x^2 + 1 \\
x^5 \bmod g(x) &= x^2 + x + 1 \\
x^4 \bmod g(x) &= x^2 + x \\
x^3 \bmod g(x) &= x + 1
\end{align*}


\[
\Rightarrow G_{\text{ht}} = 
\begin{bmatrix}
x^6 + R_1(x) \\
x^5 + R_2(x) \\
x^4 + R_3(x) \\
x^3 + R_4(x)
\end{bmatrix}
=
\begin{bmatrix}
1 & 0 & 0 & 0 & 1 & 0 & 1 \\
0 & 1 & 0 & 0 & 0 & 1 & 1 \\
0 & 0 & 1 & 0 & 1 & 0 & 1 \\
0 & 0 & 0 & 1 & 0 & 0 & 1
\end{bmatrix}
\]

\[
\Rightarrow H_{\text{ht}} = 
\begin{bmatrix}
1 & 1 & 0 & 0 & 1 & 0 & 0 \\
0 & 1 & 1 & 0 & 0 & 1 & 0 \\
1 & 0 & 1 & 0 & 0 & 0 & 1
\end{bmatrix}
\]



\newpage
\textbf{Câu hỏi 11:} Xét một mã cyclic nhị phân tuyến tính hệ thống $(9,3)$ có đa thức sinh $g(x) = 1 + x^3 + x^6$.
\begin{enumerate}
    \item[a.] Xây dựng mạch lập hệ thống cho mã theo thuật toán chia 
    \item[b.] Mô tả hoạt động của mạch, tìm từ mã đầu ra tương ứng với khối tin vào $a = 101$
    \item[c.] Kiểm tra kết quả câu b) bằng thuật toán tương ứng.
\end{enumerate}
\par\noindent\textbf{Bài làm:}
\begin{itemize}
    \item [a.] Ta có $ n=9$, $k= 3 \Rightarrow r = n-k =6; g_0 = g_3 = g_6 = 1$\\
    \begin{center}
        \includegraphics[scale=0.5]{cau11.png}
    \end{center}   
    \text {Sơ đồ mã hóa:}
    \item [b. ] Mô tả hoạt động của mạch    
    \begin{itemize}
        \item 3 nhịp đầu: Mạch  $AND$ V$_1$ mở, V$_2$ đóng, thiết bị ``1 bộ chia để tính dư'' kết thúc nhịp thứ 3, toàn bộ phần dư nằm trong 6 ô nhớ. Các dấu tuyến tính $a(x)\cdot x^\text{n-k}$ được đưa vào mạch $OR$ $H$
        \item 6 nhịp sau: dữ liệu được chia hết (phần dư). Mạch V$_1$ đóng, thiết bị $\approx $ 1 thanh ghi dịch nối tiếp. Mạch $V_2$ mở, ác dấu kiểm tra lần luowtjd được dưa ra từ bậc cao $\Rightarrow$ Bậc thấp kết thúc nhịp thứ 9, toàn bộ từ mã được đưa ra.
    \end{itemize}
    
    \text{Với a = 101}
    
    \begin{center}
    \begin{tabular}{|c|c|c|c|c|c|c|c|c|}
    \hline
    \textbf{Xung nhịp} & \textbf{Vào} & \textbf{1} & \textbf{2} & \textbf{3} & \textbf{4} & \textbf{5} & \textbf{6} & \textbf{Ra} \\
    \hline
    1 & 1 & 1 & 0 & 0 & 1 & 0 & 0 & 1 \\
    \hline
    2 & 0 & 0 & 1 & 0 & 0 & 1 & 0 & 0 \\
    \hline
    3 & 1 & 0 & 1 & 0 & 0 & 0 & 0 & 1 \\
    \hline
    4 & 0 & 0 & 1 & 0 & 1 & 1 & 0 & 1 \\
    \hline
    5 & 0 & 0 & 0 & 1 & 0 & 1 & 1 & 0 \\
    \hline
    6 & 0 & 0 & 0 & 0 & 1 & 0 & 1 & 1 \\
    \hline
    7 & 0 & 0 & 0 & 0 & 0 & 1 & 0 & 1 \\
    \hline
    8 & 0 & 0 & 0 & 0 & 0 & 1 & 1 & 0 \\
    \hline
    9 & 0 & 0 & 0 & 0 & 0 & 0 & 0 & 1 \\
    \hline
    \end{tabular}
    \end{center}
    \[
    \Rightarrow c(x) = x^8 + x^6 + x^5 + x^3 + x^2 + 1  \leftrightarrow  101101101 \]
    \item[c.] Ta lấy $c(x)$ chia cho $g(x)$ ta được: \\
    $g(x)(x^2 +1) = x^8 + x^6 + x^5 + x^3 + x^2 + 1 = c(x) $\\
    $\Rightarrow$ \text{Đúng };
    
    

\end{itemize}

\newpage
\textbf{Câu 1.13}: Nêu định nghĩa và tính chất của lượng thông tin chéo 

\textbf{Bài làm}: 
\begin{itemize}
    \item \textbf{Định nghĩa:} Lượng thông tin chéo trung bình là lượng thông tin trung bình truyền được qua kênh khi thực hiện phát và thu một tin bất kì.
    
    \[
    I(X, Y) = \sum_{k=1}^N \sum_{l=1}^M p(x_k, y_l) \log\left(\frac{p(x_k|y_l)}{p(x_k)}\right)
    \]
    
    \item \textbf{Tính chất:}
    \[
   \begin{aligned}
    I(X, Y) &= H(X) - H(X|Y) = H(Y) - H(Y|X) \\
            &= H(X) + H(Y) - H(X, Y) \\
    I(X, Y) &= I(Y, X) \\
    I(X, X) &= H(X) \\
    I(X, Y) &\geq 0,\quad I(X, Y) \leq \min\{H(X), H(Y)\}
\end{aligned}
    \]
\end{itemize}
\textbf{Câu 2.13}: Một thành phố nọ có 1\% dân số là sinh viên. Trong số sinh viên có 50\% là nam thanh niên. Số nam thanh niên trong thành phố là 32\% dân số. Giả sử ta gặp 1 nam thanh niên. Hãy tính lượng thông tin chứa trong tin khi biết rằng đó là một nam sinh viên.

\textbf{Bài làm:} Gọi biến cố xuất hiện sinh viên trong thành phố là $X$, biến cố xuất hiện nam thanh niên là $Y$.\\

Theo đề bài, ta có: $P(X) = 0{,}01$, $P(Y) = 0{,}32$.

Ta có:
\[
P(Y|X) = \frac{P(X, Y)}{P(X)} \Rightarrow P(X,Y)=0,5\cdot0,1=0,005
\]

Xác suất để gặp một nam sinh viên trong thành phố là: 
\[
P(X|Y) = \frac{P(X, Y)}{P(Y)} = \frac{0,005}{0,32} = \frac{1}{64}
\]

Lượng thông tin chứa trong tin khi biết đó là nam sinh viên là:
\[
I(X|Y) = -log(p(X|Y)) = -log(\frac{1}{64}) = 6(bit)

\textbf{Câu 4.3:} Cho sơ đồ kênh trong đó nguồn tín hiệu phát gồm \( X = \{x_1, x_2\} \). Biết xác suất phát các tín hiệu: \( p(x_1) = p(x_2) = 0{,}5 \).

\begin{figure}[H]
    \centering
    \includegraphics[width=0.6\textwidth]{4.3.png}
    \label{fig:so_do}
\end{figure}
\begin{itemize}
    \item[a)] Hãy tính \( H(X) \).
    
    \item[b)] Hãy tính \( p(X = x_n, Y = y_m) \) với \( n = 1, 2 \) và \( m = 1, 2, 3 \), từ đó tính \( H(X, Y) \) dưới dạng hàm số \( \alpha \).
    
    \item[c)] Hãy tính \( p(Y = y_m) \) với \( m = 1, 2, 3 \), từ đó tính \( H(Y) \) dưới dạng hàm số \( \alpha \).
    
    \item[d)] Hãy tính \( I(X, Y) \) dưới dạng hàm số của \( \alpha \). Hãy xác định giá trị của \( \alpha \) khi \( I(X, Y) \) đạt giá trị cực đại và khi \( I(X, Y) \) đạt giá trị cực tiểu. Hãy cho biết ý nghĩa trực quan của kênh ứng với các giá trị cực trị của \( I(X, Y) \).
\end{itemize}

\textbf{Bài làm}: 
\begin{itemize}
    \item[a)] 
    \[
    H(X) = 0{,}5 \log_2 2 + 0{,}5 \log_2 2 = 1 \text{ (bit)}
    \]

    \item[b)]
    \[
    \begin{aligned}
    &p(x_1, y_1) = 0{,}5\alpha \quad &&p(x_2, y_3) = 0{,}5(1 - \alpha)\\
    &p(x_1, y_3) = 0{,}5(1 - \alpha) \quad &&p(x_2, y_2) = 0{,}5\alpha \\
    \Rightarrow\ &H(X,Y) = -\left[ 0{,}5\alpha \log 0{,}5\alpha + 0{,}5(1 - \alpha) \log 0{,}5(1 - \alpha) \right.\\
    &\quad + 0{,}5(1 - \alpha) \log 0{,}5(1 - \alpha) + 0{,}5\alpha \log 0{,}5\alpha \Big] \\
    &= -\left[ \alpha \log \frac{\alpha}{2} + (1 - \alpha) \log \frac{1 - \alpha}{2} \right]
    \end{aligned}
    \]

    \item[c)] 
    \[
    \begin{aligned}
    &p(y_1) = p(y_1|x_1) = 0{,}5\alpha \\
    &p(y_3) = p(y_3|x_1) + p(y_3|x_2) = 0{,}5(1 - \alpha) + 0{,}5(1 - \alpha) = 1 - \alpha \\
    &p(y_2) = p(y_2|x_2) = 0{,}5\alpha \\
    \Rightarrow\ &H(Y) = -\left[ \frac{\alpha}{2} \log \frac{\alpha}{2} + (1 - \alpha) \log (1 - \alpha) + \frac{\alpha}{2} \log \frac{\alpha}{2} \right] \\
    &= -\left[ \alpha \log \frac{\alpha}{2} + (1 - \alpha) \log (1 - \alpha) \right]
    \end{aligned}
    \]

    \item[d)] 
    \[
    \begin{aligned}
    I(X, Y) &= H(X) + H(Y) - H(X, Y) \\
    &= 1 - \left[ \alpha \log \frac{\alpha}{2} + (1 - \alpha) \log (1 - \alpha) \right] \\
    &\quad + \left[ \alpha \log \frac{\alpha}{2} + (1 - \alpha) \log \frac{1}{2} \right] \\
    &= 1 + (1 - \alpha)\left[ \log \frac{1}{2} - \log(1 - \alpha) \right] \\
    &= \alpha
    \end{aligned}
    \]

    \noindent
    Khi \( I(X, Y) \) đạt cực đại \( \Leftrightarrow \alpha = 1 \) \( \Rightarrow \) Kênh không nhiễu.\\
    Khi \( I(X, Y) \) đạt cực tiểu \( \Leftrightarrow \alpha = 0 \) \( \Rightarrow \) Kênh đứt.
\end{itemize}
\textbf{Câu 2.19}: Tín hiệu thoại có băng tần W=3,4kHz.

a) Tính khả năng thông qua của kênh với điều kiện SNR=30dB.

b) Tính SNR tối thiểu cần thiết để kênh có thể truyền tín hiệu thoại số có tốc độ 4800 bps.

\textbf{Bài làm}: 
\begin{itemize}
    \item \( W = 3400\, \text{Hz} = F \)
    
    \begin{itemize}
        \item[a)] 
        \[
        \begin{aligned}
        \text{SNR} &= 10 \log_{10} \left( \frac{S}{N} \right) = 30\, \text{dB} \\
        \Rightarrow \frac{S}{N} &= 10^3 \\
        \Rightarrow C' &= F \log_2 \left(1 + \frac{S}{N} \right) = 3400 \log_2(1001) \approx 33888{,}6\, \text{bit/s}
        \end{aligned}
        \]
        
        \item[b)] 
        \[
        \begin{aligned}
        &C' = F \log_2 \left(1 + \frac{S}{N} \right) \geq 1800\, \text{bps} \\
        \Rightarrow\ &\log_2 \left(1 + \frac{S}{N} \right) \geq \frac{1800}{3400} = \frac{24}{17} \\
        \Rightarrow\ &1 + \frac{S}{N} \geq 2^{24/17} \\
        \Rightarrow\ &\frac{S}{N} \geq 2^{24/17} - 1 \\
        \Rightarrow\ &\text{SNR} \geq 10 \log_{10} \left(2^{24/17} - 1 \right) \approx 2{,}203\, \text{dB}
        \end{aligned}
        \]
    \end{itemize}
\end{itemize}
\textbf{Câu 3.21}: Cho sơ đồ kênh rời rạc không nhớ (DMC) như hình vẽ. Biết thời hạn các kí hiệu phát x_1 và x_2 đều là T_p.

\begin{figure}[H]
    \centering
    \includegraphics[width=0.6\textwidth]{4.3.png}
    \label{fig:so_do}
\end{figure}
a) Hãy tính dung lượng của kênh

b) Khảo sát sơ bộ (phác hoạ biến thiên) dung lượng kênh theo giá trị của \alpha .

c) Giải thích rõ ý nghĩa của các cực đại, cực tiểu (nếu có)

\textbf{Bài làm}: 
\begin{itemize}
    \item[a)] Dung lượng kênh: 
    \[
    C = \max_X I(X,Y) = \max_X \left[ H(Y) - H(Y|X) \right]
    \]
    Có:
    \[
    H(Y|X) = - \sum_{i=1}^{2} \sum_{j=1}^{3} p(x_i, y_j) \log p(y_j | x_i)
    \]
    \[
    = - \left[ p_0 \alpha \log \alpha + p_0 (1 - \alpha) \log (1 - \alpha) + (1 - p_0)(1 - \alpha) \log (1 - \alpha) + (1 - p_0) \alpha \log \alpha \right]
    \]
    \[
    = - \left[ \alpha \log \alpha + (1 - \alpha) \log (1 - \alpha) \right]
    \Rightarrow H(Y|X) \text{ phụ thuộc } X
    \]
    \[
    \Rightarrow I(X,Y) \text{ max } \Leftrightarrow H(Y) \text{ max}
    \]
    Có:
    \[
    p(y_1) = p(y_1  x_1) = p_0 \alpha
    \]
    \[
    p(y_3) = p(y_3  x_1) + p(y_3  x_2) = p_0(1 - \alpha) + (1 - p_0)(1 - \alpha) = 1 - \alpha
    \]
    \[
    p(y_2) = (1 - p_0)\alpha
    \]
    Mà $p(y_3)$ không phụ thuộc vào $X$, $p(y_1) = p(y_2) = \frac{\alpha}{2} \Rightarrow p_0 = \frac{1}{2}$

    \[
    \Rightarrow H(Y) = - \left[ \frac{\alpha}{2} \log \frac{\alpha}{2} + (1 - \alpha) \log (1 - \alpha) + \frac{\alpha}{2} \log \frac{\alpha}{2} \right]
    \]
    \[
    = - \left[ \alpha \log \frac{\alpha}{2} + (1 - \alpha) \log (1 - \alpha) \right]
    \]

    \[
    \Rightarrow C = \max_X I(X,Y) = - \left[ \alpha \log \frac{\alpha}{2} + (1 - \alpha) \log (1 - \alpha) \right] 
    + \left[ \alpha \log \alpha + (1 - \alpha) \log (1 - \alpha) \right]
    \]

    \[
    = \alpha (\log \alpha - \log \frac{\alpha}{2}) = \alpha \log 2 = \alpha
    \]
    \noindent b) 
\begin{figure}[H]
    \centering
    \includegraphics[width=0.4\textwidth]{3.21.png}
    \label{fig:so_do}
\end{figure}
    \[
    C = 0 \Leftrightarrow \alpha = 0 \quad \text{(Kênh hỏng)}
    \]
    \[
    C = 1 \Leftrightarrow \alpha = 1 \quad \text{(Kênh không nhiễu)}
    \]
\end{itemize}
\textbf{Câu 2.9}: 

a) Hãy cho biết nhược điểm của mã Huffman khi sử dụng cho mục đích nén dữ liệu

b) Cho hai bộ mã khác nhau được sử dụng để mã hóa các ký tự $a$, $b$, $c$, $d$. Trong bảng, $p_i$ là xác suất xuất hiện của mỗi ký tự. Hãy tính chiều dài trung bình khi mã hóa một ký tự đối với mỗi bộ mã.

\begin{table}[H]
\centering
\renewcommand{\arraystretch}{2.0} 
\setlength{\tabcolsep}{12pt} 
\begin{tabular}{|c|c|c|c|}
\hline
\textbf{$a_i$} & \textbf{$c_1(a_i)$} & \textbf{$c_2(a_i)$} & \textbf{$p_i$} \\ 
\hline
a & 1000 & 0   & $\dfrac{1}{2}$ \\
b & 0100 & 10  & $\dfrac{1}{4}$ \\
c & 0010 & 110 & $\dfrac{1}{8}$ \\
d & 0001 & 111 & $\dfrac{1}{8}$ \\
\hline
\end{tabular}
\label{tab:mahoa_dep}
\end{table}

\textbf{Bài làm}: 
\begin{itemize}
    \item[a)] Nhược điểm của mã Huffman khi sử dụng để nén dữ liệu:
    \begin{itemize}
        \item Tập các từ mã tối ưu không là duy nhất.
        \item Bên nhận muốn giải mã thì phải có bảng mã giống bên gửi.
    \end{itemize}
    $\Rightarrow$ Khi nén các tập tin bé, hệ số nén không cao.
    
    \item[b)] 
    \begin{align*}
        \bar{n}_{c_1} &= n_{c_1}(a) \cdot p(a) + n_{c_1}(b) \cdot p(b) + n_{c_1}(c) \cdot p(c) + n_{c_1}(d) \cdot p(d) \\
        &= 4 \cdot \dfrac{1}{2} + 4 \cdot \dfrac{1}{4} + 4 \cdot \dfrac{1}{8} + 4 \cdot \dfrac{1}{8} = 4 \quad \text{(dấu)} \\
        \bar{n}_{c_2} &= n_{c_2}(a) \cdot p(a) + n_{c_2}(b) \cdot p(b) + n_{c_2}(c) \cdot p(c) + n_{c_2}(d) \cdot p(d) \\
        &= 1 \cdot \dfrac{1}{2} + 2 \cdot \dfrac{1}{4} + 3 \cdot \dfrac{1}{8} + 3 \cdot \dfrac{1}{8} = 1{,}75 \quad \text{(dấu)}
    \end{align*}
\end{itemize}
\textbf{Câu 3.11:} Hãy thực hiện mã hoá Huffman cho nguồn rời rạc \( A \) sau:

\[
A = 
\begin{pmatrix}
a_1 & a_2 & a_3 & a_4 & a_5 & a_6 & a_7 & a_8 & a_9 & a_{10} \\
\frac{1}{2} & \frac{1}{8} & \frac{1}{8} & \frac{1}{8} & \frac{1}{32} & \frac{1}{32} & \frac{1}{32} & \frac{1}{64} & \frac{1}{128} & \frac{1}{128}
\end{pmatrix}
\]

\vspace{0.5em}

\textbf{Yêu cầu:}
\begin{itemize}
    \item Thực hiện mã hoá Huffman cho nguồn trên.
    \item Đánh giá hiệu quả của phép mã hoá.
    \item Giải mã cho dãy bit nhận được có dạng: \texttt{1 0 1 1 0 0 1 1 1 0 1 0 1}.
\end{itemize}

\textbf{Bài làm}: 
\begin{figure}[H]
    \centering
    \includegraphics[width=0.9\textwidth]{3.11.png}
    \label{fig:so_do}
\end{figure}
\textbf{Mã hoá Huffman}

\begin{center}
\begin{tabular}{cccccccccc}
$a_1$ & $a_2$ & $a_3$ & $a_4$ & $a_5$ & $a_6$ & $a_7$ & $a_8$ & $a_9$ & $a_{10}$ \\
1     & 011   & 010   & 001   & 00011 & 00010 & 00001 & 000001 & 0000001 & 0000000 \\
\end{tabular}
\end{center}

\[
\bar{n} = \frac{1}{2} + \frac{3}{8} + \frac{3}{8} + \frac{3}{8} + \frac{5}{32} + \frac{5}{32} + \frac{5}{32} + \frac{6}{64} + \frac{7}{128} + \frac{7}{128} = \frac{147}{64} = 2.296875 \text{ (dấu)}
\]

\[
H(A) = \frac{1}{2} \log 2 + \frac{1}{8} \log 8 \times 3 + \frac{3}{32} \log 32 + \frac{1}{64} \log 64 + \frac{2}{128} \log 128
\]

\[
\Rightarrow H(A) = 2.296875 \text{ (bit)}
\]

\[
\Rightarrow \frac{H(A)}{\bar{n}} = 1 \Rightarrow \text{Mã hoá tối ưu}
\]

\textbf{Giải mã:} \texttt{1011001110101...} $\Rightarrow a_1, a_2, a_3, a_9, a_1, a_2, a_1...$\\

\textbf{Câu 1.1}: 
a) Viết biểu thức tính lượng tin chứa trong tin $x$ với xác suất $p(x)$.

b) Cho $p(x) = \dfrac{1}{16}$. Tính lượng tin riêng chứa trong sự kiện $x$.\\

\textbf{Bài làm}:

a) Biểu thức tính lượng tin chứa trong tin $x$ với xác suất $p(x)$:

\[
I(x) = -\log_2 p(x)
\]

b) Lượng tin riêng chứa trong sự kiện $x$:

\[
I(x) = -\log_2 \left( \dfrac{1}{16} \right) = \log_2 16 = 4 \text{ (bit)}
\]\\

\textbf{Câu 3.3}: Cho mã khối tuyến tính (7,3) với ma trận sinh:\\

\quad G_{3 \times 7}= \[
\begin{pmatrix}
1 & 0 & 0 & 1 & 1 & 1 & 0 \\
0 & 1 & 0 & 0 & 1 & 1 & 1 \\
0 & 0 & 1 & 1 & 0 & 1 & 1
\end{pmatrix}
\]\\

a) Tìm ma trận kiểm tra H cho bộ mã

b) Tìm khoảng cách Hamming của bộ mã

c) Cho bản tin đầu vào m=110, tìm từ mã tương ứng\\

\textbf{Bài làm}: 

a)\quad 
\[
G_{3 \times 7} = 
\begin{bmatrix}
1 & 0 & 0 & 0 & 1 & 1 & 0 \\
0 & 1 & 0 & 0 & 0 & 1 & 1 \\
0 & 0 & 1 & 0 & 1 & 0 & 1 \\
\end{bmatrix}
= \left[ I_3 \ \big| \ P \right]
\]

\[
\Rightarrow H_{4 \times 7} = \left[ P^T \ \big| \ I_4 \right]
= 
\begin{bmatrix}
1 & 0 & 1 & 1 & 0 & 0 & 0 \\
1 & 1 & 0 & 0 & 1 & 0 & 0 \\
1 & 1 & 1 & 0 & 0 & 1 & 0 \\
0 & 1 & 1 & 0 & 0 & 0 & 1 \\
\end{bmatrix}
\]

b) Có các cột 1, 4, 5, 6 phụ thuộc tuyến tính \( \Rightarrow \) d_0 = 4.

c) \( m = 110 \Rightarrow m = (1,1,0) \)

Ta có: \( c = m \cdot G \)

\[
(c_1, c_2, c_3, c_4, c_5, c_6, c_7) 
= (1, 1, 0) \cdot 
\begin{bmatrix}
1 & 0 & 0 & 0 & 1 & 1 & 0 \\
0 & 1 & 0 & 0 & 0 & 1 & 1 \\
0 & 0 & 1 & 0 & 1 & 0 & 1 \\
\end{bmatrix}
\]

\[
\Rightarrow \begin{cases}
c_1 = 1 \cdot 1 + 1 \cdot 0 + 0 \cdot 0 = 1 \\
c_2 = 1 \\
c_3 = 0 \\
c_4 = 1 \\
c_5 = 0 \\
c_6 = 0 \\
c_7 = 1 \\
\end{cases}
\Rightarrow \mathbf{c} = (1, 1, 0, 1, 0, 0, 1)
\]\\

\textbf{Câu 1.6}: Chứng minh rằng nếu $g(x)$ là đa thức sinh của một mã cyclic (n,k) bất kì thì hệ số tự do của g_0=1. \\

\textbf{Bài làm}: 
Giả sử $g_0 = 0$.

Ta có:
\[
g(x) = g_1 x + g_2 x^2 + \dots + g_{r-1} x^{r-1} + x^r \quad (g_r = 1 \text{ vì } \deg g(x) = r)
\]

\[
= x \cdot (g_1 + g_2 x + \dots + g_{r-1} x^{r-2} + x^{r-1})
\]

Đặt:
\[
f(x) = g_1 + g_2 x + \dots + g_{r-1} x^{r-2} + x^{r-1} = \dfrac{g(x)}{x}
\]

Từ mã tương ứng của $f(x)$ là kết quả của từ mã tương ứng $g(x)$ dịch trái 1 bít.  
$g(x)$ là đa thức sinh \\  
$\Rightarrow f(x)$ là đa thức mã.

Có 
\[
\begin{cases}
$\deg f(x) = \deg g(x) - 1 = r - 1 < \deg g(x)$\\
f(x) \neq 0 \\
\end{cases}
\Rightarrow \text{mâu thuẫn với } \text{định nghĩa } g(x).
\]

\textbf{Vậy} giả sử sai $\Rightarrow g_0 = 1$ (đpcm).

\textbf{Câu 3.1}: \\
a) Cho mã khối tuyến tính (n,k) có khoảng cách  tối thiểu Hamming $d_0=8$. Hỏi mã này có khả năng phát hiện bao nhiêu sai và sửa bao nhiêu sai?\\
b) Một mã khối tuyến tính (n,2) có khoảng cách tối thiểu Hamming $d_0=5$. Xác định chiều dài n tối thiểu.\\
c) Cho biết có tồn tại một bộ mã khối tuyến tính với các tham số n=15, k=7, $d_{min}=5$ hay không?

\textbf{Bài làm}: 
\begin{enumerate}[a)]
    \item $d_0 = 8$.
    
    $\Rightarrow$ Khả năng phát hiện lỗi $t= d_0 - 1 = 7$ lỗi sai.\\
    Khả năng sửa lỗi $t \le \left\lfloor \dfrac{d_0 - 1}{2} \right\rfloor = \left\lfloor \dfrac{7}{2} \right\rfloor = 3$ lỗi sai.

    \item $d_0 = 5$, $k = 2$ (Giới hạn Griesmer):

    \[
    n \ge \sum_{i=0}^{k-1} \left\lceil \dfrac{d_0}{2^i} \right\rceil = \left\lceil \dfrac{5}{1} \right\rceil + \left\lceil \dfrac{5}{2} \right\rceil = 5 + 3 = 8 \text{ (dấu)}
    \]

    \item 
    Giới hạn Griesmer:

    \[
    n \ge \sum_{i=0}^{k-1} \left\lceil \dfrac{d_0}{2^i} \right\rceil
    \]

    \[
    \Rightarrow 15 \ge \left\lceil \dfrac{5}{1} \right\rceil + \left\lceil \dfrac{5}{2} \right\rceil + \left\lceil \dfrac{5}{4} \right\rceil + \left\lceil \dfrac{5}{8} \right\rceil + \left\lceil \dfrac{5}{16} \right\rceil + \left\lceil \dfrac{5}{32} \right\rceil + \left\lceil \dfrac{5}{64} \right\rceil
    \]

    \[
    = 5 + 3 + 2 + 1 + 1 + 1 + 1 = 14 \le 15 \quad \text{(thoả mãn)}
    \]

    $\Rightarrow$ Vậy tồn tại mã tuyến tính $(15;7;5)$.
\end{enumerate}


\newpage
\textbf{Câu x}: Nội dung câu hỏi

\textbf{Bài làm}: Nội dung bài làm câu x
%% 

%% =================
%%		Nội dung bài làm sẽ kết thúc ở đây
%% =================


\end{document}